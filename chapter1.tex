\section{Основные определения необходимые для изучения протоколов мониторинговой реконструкции в рамках задачи бинаризации и обзор литературы по задаче сегментации}

В рамках текущей главу будут сформулированы основные определения, необходимые в исследовании протоколов мониторинговой реконструкции в рамках задачи бинаризаци. Также будет проведен литературный обзор алгоритмов бинаризации.

\subsection{Томография под контролем реконструекции}

Компьютерная томография — это один из наиболее эффективных неразрушающих методов исследования внутренней структуры объектов.  В процессе компьютерной томографии объект просвечивается рентгеновским излучением под разными углами, а ослабленное излучение регистрируется детектором. Каждое такое измерение называется проекцией. При достаточном количестве проекций возможно воссоздать объёмное изображение внутренней структуры объекта.

Метод компьютерной томографии широко применяется в медицине, промышленности, материаловедении и других областях, где требуется неразрушающее исследование внутренней структуры. Он позволяет выявлять дефекты, измерять геометрию, строить цифровые модели и анализировать многослойные объекты. Точность и универсальность метода делают его важным инструментом как в научных исследованиях, так и в прикладных задачах.

Одной из наиболее актуальных проблем компьютерной томографии является доза излучения, которой подвергается объект в процессе исследования. Многие материалы в индустриальной томографии достаточно чувствительны к излучению и могут охрупчать или разрушиться в последствии сканирования. В медицине эта проблема не менее актуальна, так как превышение минимально допустимой дозы излучения сопряжено с рисками для здоровья.

На основе этой проблемы сформулирована одна из наиболее актуальных задач в компьютерной томографии - снижение дозы излучения, которой подвергается объект исследования, по принципу ALARA (As Low As Reasonably Achievable) \cite{slovis2002alara}.

Другой связанной задачей в компьютерной томографии является снижение времени, необходимого для исследования объекта. Эта проблема особенно актуальна в индустриальной томографии, где такой процесс может длиться более часа.

Одним из подходов к решению сформулированных задач является томография под контролем реконструкции, также называемая мониторинговой реконтсрукцией. 

В таком подходе проекционные данные собираются постепенно и после каждой новой итерации выполняется реконструкция по имеющимся проекциям. Далее оценивается качество полученной реконструкции. В случае удовлетворительного результата реконструкции процесс сканирования может быть остановлен досрочно. 

Таким образом томография под контролем реконструкции может рассматриваться как anytime-процесс, при котором результат постепенно улучшается по мере накопления проекций, и на любом этапе можно принять решение об остановке \cite{bulatov2020monitored}. Такой подход является уместным когда цена вычислений, в данном случае доза облучения и количество времени на исследование, сопоставимы с ценой неточностей и ошибок в реконструкции.

В рамках данной работы будет исследован процесс томографии под контролем реконструкции в рамках задачи бинарной сегментации реконструкции. Во многих прикладных задачах конечной целью томографии является не визуализация самой реконструкции, а получение бинарной маски для построения CAD-модели, обнаружения дефектов и других задач, требующих выделения объекта исследования и его дальнейшей обработки.

\subsection{Основные определения необходимые для исследования протоколов мониторинговой реконструкции в рамках задачи бинаризации}

Базовым объектом, изучаемом в рамках текущей работы, является изображение.

Изображением размера \(n=(n_1, n_2, n_3)\) будем называть отображение \(I_n^m : P_3 \rightarrow G\), в котором элементы области определения \(P \subset Z^3\)  называются пикселями (вокселями), а \(n_i \in N, i = 1, 2, 3\) и \((G, +)\) - аддитивная абелева группа \cite{NikolaevPhdthesis}.

Формально задачу бинаризации, также называемую бинарной сегментацией, можно поставить следующим образом \cite{fu1981survey}:
Дано изображение \( I_n^m \) и предикат однородности \(H : P \rightarrow {0, 1}\). Найти такие два конечных, непересекающихся подмножества \( P_1, P_2\) множества \(P\), которые удовлетворяют следующим условиям: 
\begin{enumerate}
    \item Объеденение множеств \(P_1\) и \(P_2\) является исходным множеством \(P\): \( P_1 \cup  P_2 = P \) 
    \item Множества \(P_1\) и \(P_2\) являются однородными по предикату \(H\): \(H(P_1) = H(P_2) = 1\)
    \item Объеденение множеств \(P_1\) и \(P_2\) не однородно: \(H(P_1 =cup P_2) = 0\)
\end{enumerate}

Результатом решения задачи бинаризации является бинарной маской изображения \(\image\).
Бинарной маской изображения \(I_n^m\) будем называть отображение \(\tilde{I_n^m} : P_3 \rightarrow \{0, 1\}\), множество определения которого совпадает с множеством определения \(I_n^m\).

Множеством объекта назовем множество пикселей, при которых бинарная маская изображения принимает значение 1: \(\{x \in P | \tilde{I_n^m}(x) = 1 \}\).

Далее проведем литературный обзор и рассмотрим существующие алгоритмы бинаризации.



\subsection{Литературный обзор существующих классов алгоритмов бинаризации и определение их применимости в рамках исследования мониторинговой реконструкции в рамках задачи бинаризации}

В рамках данной работы рассматривается поведение сегментации изображения реконструкции в процессе томографии под контролем реконструкции. При этом выбор алгоритмов сегментации является важным шагом для дальнейшего проведения экспериментов.

С целью выбора алгоритма необходимо провести обзор доступных алгормитмов сегментации.

В научной перидике доступно большое количество публикаций на тему алгоритмов сегментации. Исследование \cite{zhang2006advances} показало устойчивый рост количества ежегодно предложенных алгоритмов с 1995 года по 2006 года.

Систематизация и классификация доступных алгоритмов необходима для выбора наиболее подходящих для проведения экспериментов. 

Хорошей отправной точкой в изучении доступных алгоритмов сегментации является рассмотрение схем их классификации. 

Существует множество различных схем классификации алгоритмов сегментации. Например, часто используется классификация по одному признаку, такому как способ обработки изображения или степень участия человека в процессе сегментации \cite{wirjadi2007survey}.

В работе \cite{ханыков2018классификация} также предложена схема обобщённой классификации, объединяющая несколько одно-признаковых подходов в единую структуру.

В рамках проведённого литературного обзора поиск алгоритмов сегментации был проведен на основе их классификации по принципу работы.

Первым классом алгоритмов, встречающихся в литературе, является семейство пороговых алгоритмов. 

Пороговые алгоритмы бинаризации выполняют классификацию вокселей изображения на основе заданного порогового значения интенсивности. Воксели с интенсивностью ниже порога относятся к фону, тогда как воксели с интенсивностью выше или равной порогу классифицируются как принадлежащие объекту.

Существует большое количество разнообразных пороговых алгоритмов, основыные из которых изложены в работы \cite{wirjadi2007survey}. 

На пример, часть алгоритмов устанавливают порог глобавально, другие - устанавливают его индивидуально для каждого пикселя. 

Другим широко используемым подходом к сегментации является метод роста области (region growing), основанный на объединении вокселей с близкими характеристиками по мере распространения от заданных начальных точек \cite{adams1994seeded}.

Этот метод является полуавтоматическим, то есть он требует участия пользователя в процессе своей работы. В частности требуется ввести набор начальных точек.

В литературе встречается множество модификаций этого метода, включая автоматический вариант алгоритма, не требующий указания начальной точки \cite{lin2000unseeded}. 

Несмотря на такое множество модификаций, метод роста области достаточно трудно реализуем в контексте задачи сегментации под контролем реконструкции.

Полуавтоматический характер метода и ресурсоёмкость его автоматической модификации делают его непрактичным для использования в рамках данного исследования.

Следующим направлением, широко представленным в литературе, являются методы кластеризации. Схожесть формальных постановок задач сегментации и кластеризации способствует применению кластеризационных алгоритмов в контексте сегментации.

В частности, алгоритмы K-means \cite{sarker2017segmentation} и mean shift \cite{comaniciu2002mean} нередко применяются при решении задач сегментации.

Применимость алгоритмов кластеризации к сегментации изображений реконструкции требует дополнительного анализа, поскольку реализация их вычислительно эффективных версий представляет собой нетривиальную задачу.

Представленные ранее классы алгоритмов во многом однородны по своей структуре и принципам работы. В то время как последующие группы объединяют существенно более разнородные методы, классифицированные по более общим признакам.

Такой группой алгоритмов являются методы на базе теории графов \cite{camilus2012review}. 

Принципы работы этих методов достаточно разнообразны. Некоторые работы применяют алгоритмы поиска разрезов в графе \cite{boykov2003computing, peng2019interactive}, другие работы адаптируют алгоритм поиска максимального пока под задачу сегментации \cite{zeng2008topology}.

Также встречаются и нестандартные подходы к сегментации в этой группе алгоритмов.

Один из таких нестандартных подходов предложен в работе \cite{felzenszwalb2004efficient}, где реализован алгоритм, концептуально близкий к Unseeded Region Growing, но основанный на представлении изображения в виде взвешенного графа.

Следующей группой являются вероятностные алгоритмы. Принцип работы методов в этой группе основан на некотором априорном предположении о распределении значений пикселей в рамках изображения.

На пример методы, изложенные в работах \cite{hu2003volumetric, ayed2006unsupervised} основаны на предположении, что значение пикселей объектов в  изображении имеет распределение вейбюля.

Последней крупной группой алгоритмов, встречающейся в литературе, это методы на основе машинного обучения.

К алгоритмам на основе машинного обучения относятся как нейросетевые методы, применяемые в общей задаче сегментации изображений \cite{lu20193d, ха2016свёрточная}, так и подходы, адаптированные под изображения реконструкции в рамках томографии \cite{milletari2016v}. 

Также встречаются менее распространённые методы, включая клеточные нейронные сети \cite{liu2011industrial} и отдельные примеры полуавтоматических алгоритмов, использующих классические модели, такие как метод опорных векторов \cite{lang2022ai, gonella2019semi}.

Некоторые выявленные в литературе методы не вписываются в описанные выше категории и базируются на оригинальных, зачастую уникальных подходах. 

Алгоритмы этого типа представлены в небольшом числе работ и не формируют обособленного направления.

Одним из таких алгоритмов является Полуавтоматический SegMo \cite{nagai2019segmo}, разработаный для сегментации изображений реконструкции индустриальной томографии.

Сам алгоритм достаточно сложный в реализации и требует активного участия пользователя, соответственно не подходит для исследования томографии под контролем реконструкции, однако авторы заявляют достаточно высокое качество сегментации на выходе.

Среди редких методов также встречаются алгоритмы на основе множеств уровня, использующие эволюцию поверхностей для выделения объектов в объёме. В работе \cite{farag20043d} такой подход применён для трёхмерной сегментации сосудистой системы на данных магнитно-резонансной ангиографии.

В рамках литературного обзора были изучены пороговые алгоритмы, методы роста области, алгоритмы на основе кластеризации, теории графов, вероятностные подходы, а также методы, использующие машинное обучение и нестандартные оригинальные принципы.

Из большого разнообразия доступных алгоритмов была выбрана группа пороговых методов.

Причина выбора пороговых методов заключается в их высокой степени изученности. В литературе представлено множество работ, посвящённых их модификации, анализу и практическому применению, обладающих высокой цитируемостью.

Применение алгоритмов других групп в контексте томографии под контролем реконструкции сопряжено с рядом ограничений. 

Многие из них являются полуавтоматическими и требуют участия пользователя, что делает их непригодными для включения в итеративный процесс реконструкции. 

Классы алгоритмов с высокой вычислительной сложностью, такие как методы кластеризации и алгоритмы на основе множеств уровня, затруднительно использовать в силу необходимости их повторного запуска на каждой итерации томографического сканирования. 

Методы, основанные на нейронных сетях, требуют большого объёма размеченных данных, которые на текущий момент недоступны для рассматриваемой задачи.

Следующим этапом после выбора класса является формирование перечня конкретных алгоритмов, из которых будет производиться отбор для последующего эксперимента.

Следующая глава будет посвещена сбору необходимых материалов для конвейера экспериментов и его реализации.