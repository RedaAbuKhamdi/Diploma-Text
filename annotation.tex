\section*{Аннотация}

В работе рассматривается задача оптимизации процедуры компьютерной томографии с использованием протоколов мониторинговой реконструкции. Проведено исследование применения правила останова, основанного на анализе бинаризации промежуточных реконструкций. Для оценки качества использовались метрики IOU и Symmetric Boundary DICE, а в качестве алгоритмов бинаризации применялись методы Отсу, классического порогования и аффинного Ниблэка. Результаты экспериментов показали, что предложенное правило позволяет в среднем вдвое сократить количество необходимых углов проекций без значимой потери качества бинаризации реконструкции, что открывает перспективы для снижения дозовой нагрузки и времени сканирования.