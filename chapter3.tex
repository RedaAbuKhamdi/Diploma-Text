\section{Глава 3}

\subsection{Правило останова}

Томография под контролем реконструкции является итеративным процессов, соответственно в каждой итерации необходимо принять решение об остановке процесса на текущей реконструкции или добавления большего количества проекций.

В рамках текущей работы изучается автоматическое принятие такогого решение заранее сформулированным правилом останова на основе анализа бинаризации.

Для изучения данного вопроса требуется сформулировать такое правило останова и рассмотреть его эффект на среднее качество бинаризации по изображениям на количество углов с помощью реализованного конвейера экспериментов.

В реальных условиях применения правила останова в процессе томографии под контролем реконструкции отсутствует доступ к эталонным данным, соответственно правило останова, основанное на объективной оценке качества, является неприменимым в данном контексте.

Однако собранный для конвейера экспериментов набор метрик позволяет всесторонне оценить схожесть двух входных изображений.

В процессе томографии под контролем реконструкции как правило в первой же итерации, с набором из 4 углов, маловероятно получить достаточно качественное изображения для остановки процесса. 

Соответственно, для формулировки осмысленного правила останова требуется сравнение двух реконструкций — полученной на текущем наборе углов и реконструкции, соответствующей предыдущему набору. 

Предполагается, что на каждой итерации доступны изображения \((\image)_{i-1}\) и \((\image)_i\), где \(i = 1, ..., n\) это индекс набора углов, а реконструкция, полученная на первом наборе, имеет индекс \(0\).

Правилом останова назовем отображение \(S : R \rightarrow \{0, 1\}\), где значение 0 означает продолжение процесса томографии под контролем реконструкции, а 1 - завершение процесса на текущей реконструкции.

Сравнивая бинаризации двух соседних реконструкций можно получить информацию о степени изменения бинаризации с добавлением большего количества проекций.

Аргументом для правила останова будет являться значение метрики \(M\) на бинаризациях \(\hat{(\image)} _{i-1}\) и \(\hat{(\image)}_i\). 

Так как объективная оценка качества бинаризации не доступно при приминении правило останова, его формулировка требует некоторых предположений и эвристики.

Первым предположением, лежащим в основе формулировки правила останова, является допущение о том, что при достаточно высокой степени схожести бинаризаций двух соседних реконструкций добавление новых проекций не приводит к значимым изменениям результата. Формально это выражается условием:
\begin{equation} \label{eq:stoppingv1}
    S(\hat{(\image)}_{i-1}, \hat{(\image)}_{i}) = \begin{cases}
        1, M(\hat{(\image)}_{i-1}, \hat{(\image)}_i) \geq c\\
        0, M(\hat{(\image)}_{i-1}, \hat{(\image)}_i) < c

    \end{cases}
\end{equation}
где \(M\) — выбранная метрика сравнения бинарных масок, а \(c = const \in [0, 1]\) — фиксированное константное пороговое значение.

Описанное правило уже позволяет получить хороший результат в случае когда качество реконструекции с увеличением количества углов плавно улучшается.

Примерами таких изображений в наборе данных конвейера экспериментов являются объёмы "Кролик" и "Статуэтка".

В случаях, когда наблюдается резкий скачок качества при включении проекций под углами, совпадающими с основными структурными элементами исследуемого объекта, правило останова, описанное в формуле~\eqref{eq:stoppingv1}, может привести к слишком раннему завершению процесса. Это связано с тем, что до момента скачка несколько соседних реконструкций могут оказаться достаточно схожими между собой, несмотря на то, что качество ещё не достигло оптимального уровня.

Такими свойствами в наборе данных обладают изображения "Решётка" и "Наклонная решётка". В первом случае, благодаря ортогональной структуре плоскостей, приемлемое качество достигается уже при первом наборе углов. Во втором случае качество улучшается скачкообразно — при добавлении углов, совпадающих с наклонами отдельных семейств плоскостей.

Такие ситуации можно учесть в правиле останова добавив дополнительное условие на минимальное количество углов в наборе. 

Хотя в текущем виде правило останова не может сработать на самом первом наборе углов — для расчёта метрики требуется как минимум две реконструкции — этого недостаточно для устранения проблемы преждевременной остановки. Чтобы её избежать, необходимо дополнительно ограничить минимальный индекс, с которого правило начинает применяться.

