\section{Глава 3}

\subsection{Правило останова}

Томография под контролем реконструкции является итеративным процессов, соответственно в каждой итерации необходимо принять решение об остановке процесса на текущей реконструкции или добавления большего количества проекций.

В рамках текущей работы изучается автоматическое принятие такогого решение заранее сформулированным правилом останова на основе анализа бинаризации.

Для изучения данного вопроса требуется сформулировать такое правило останова и рассмотреть его эффект на среднее качество бинаризации по изображениям на количество углов с помощью реализованного конвейера экспериментов.

В реальных условиях применения правила останова в процессе томографии под контролем реконструкции отсутствует доступ к эталонным данным, соответственно правило останова, основанное на объективной оценке качества, является неприменимым в данном контексте.

Однако собранный для конвейера экспериментов набор метрик позволяет всесторонне оценить схожесть двух входных изображений.

В процессе томографии под контролем реконструкции как правило в первой же итерации, с набором из 4 углов, маловероятно получить достаточно качественное изображения для остановки процесса. 

Соответственно, для формулировки осмысленного правила останова требуется сравнение двух реконструкций — полученной на текущем наборе углов и реконструкции, соответствующей предыдущему набору. 

Предполагается, что на каждой итерации доступны изображения \((\image)_{i-1}\) и \((\image)_i\), где \(i = 1, ..., n\) это индекс набора углов, а реконструкция, полученная на первом наборе, имеет индекс \(0\).

Правилом останова назовем отображение \(S : R \rightarrow \{0, 1\}\), где значение 0 означает продолжение процесса томографии под контролем реконструкции, а 1 - завершение процесса на текущей реконструкции.

Сравнивая бинаризации двух соседних реконструкций можно получить информацию о степени изменения бинаризации с добавлением большего количества проекций.

Аргументом для правила останова будет являться значение метрики \(M\) на бинаризациях \(\hat{(\image)} _{i-1}\) и \(\hat{(\image)}_i\). 

Так как объективная оценка качества бинаризации не доступно при приминении правило останова, его формулировка требует некоторых предположений и эвристики.

Первым предположением, лежащим в основе формулировки правила останова, является допущение о том, что при достаточно высокой степени схожести бинаризаций двух соседних реконструкций добавление новых проекций не приводит к значимым изменениям результата. Формально это выражается условием:
\begin{equation} \label{eq:stoppingv1}
    S(\hat{\image}_{i-1}, \hat{\image}_{i}) =
    \begin{cases}
        1, & \text{если } M((\hat{\image})_{i-1}, (\hat{\image})_i) \geq c \\
        0, & \text{иначе}
    \end{cases}
\end{equation}
где \(M\) — выбранная метрика сравнения бинарных масок, а \(c = const \in [0, 1]\) — фиксированное константное пороговое значение.

Описанное правило уже позволяет получить хороший результат в случае когда качество реконструекции с увеличением количества углов плавно улучшается.

Примерами таких изображений в наборе данных конвейера экспериментов являются объёмы "Кролик" и "Статуэтка".

В случаях, когда наблюдается резкий скачок качества при включении проекций под углами, совпадающими с основными структурными элементами исследуемого объекта, правило останова, описанное в формуле~\eqref{eq:stoppingv1}, может привести к слишком раннему завершению процесса. Это связано с тем, что до момента скачка несколько соседних реконструкций могут оказаться достаточно схожими между собой, несмотря на то, что качество ещё не достигло оптимального уровня.

Такими свойствами в наборе данных обладают изображения "Решётка" и "Наклонная решётка". В первом случае, благодаря ортогональной структуре плоскостей, приемлемое качество достигается уже при первом наборе углов. Во втором случае качество улучшается скачкообразно — при добавлении углов, совпадающих с наклонами отдельных семейств плоскостей.

Такие ситуации можно учесть в правиле останова добавив дополнительное условие на минимальное количество углов в наборе. 

Хотя в текущем виде правило останова не может сработать на самом первом наборе углов — для расчёта метрики требуется как минимум две реконструкции — этого недостаточно для устранения проблемы преждевременной остановки. Чтобы её избежать, необходимо дополнительно ограничить минимальный индекс, с которого правило начинает применяться.

Модифицируем формулу \eqref{eq:stoppingv1} - добавим в нее условие останова с определенного индекса:

\begin{equation} \label{eq:stoppingv2}
    S(\hat{\image}_{i-1}, \hat{\image}_{i}) =
    \begin{cases}
        1, & \text{если } M((\hat{\image})_{i-1}, (\hat{\image})_i) \geq c \text{ и } i \geq \alpha \\
        0, & \text{иначе}
    \end{cases}
\end{equation}
где \(\alpha = \text{const}\) — минимальный индекс, начиная с которого применяется правило останова.

Такая модификация позволяет снизить вероятность останова на локальном максимуме.

Сформулированное в формуле \eqref{eq:stoppingv2} правило останова носит предварительный характер, однако его достаточно для изучения вопроса возможности применения правил останова в томографии под контролем реконструкции на базе их бинаризаций.

Далее спроектируем эксперименты для исследования эффективности правила останова.

\subsection{Эксперименты над правилом останова и их результаты}

Модуль оценки, в режиме соседних бинаризаций, реализованного конвейера экспериментов уже содержит значения метрик по различию бинаризаций реконструкций с соседними наборами углов. Соответственно все необходимые данные для изучения эффекта правила останова уже включены в результаты конвейера экспериментов.

Эксперименты с правилом останова заключаются в переборе его параметров - минимального индекса угла \(\alpha\) и порога схожести соседних бинаризаций \(c\).

Так как данные уже доступны из результатов предыдущих модулей конвейера экспериментов, для исследования правила останова не требкется изменения текущего исходного кода - достаточно в модуле визуализации добавить новую компоненту в директорию visualizators.

Модуль визуализации обеспечивает доступ ко всем экспериментальным данным через единый интерфейс ResultData и передает его в качестве аргумента всем компонентам.

Реализация экспериментов с првилом останова предполагает симуляцию его применения в процессе томографии под контролем реконструкции.

Процедура проведения экспериментов с правилом останова описана в рисунке \ref*{fig:procedurestoppingrule}.

\gostfigure{procedurestoppingrule}{Процедура проведения экспериментов по исследованию эффективности правила останова; 1 - Блок рассчета среднего по набору данных значение угла и метрики при применении правила останова; 2 - Блок добавления контроля - значений углов и метрик без применения правила останова}{procedurestoppingrule}

Процедура принимает восемь аргументов: объект ResultData, 
имя метрики сравнения бинарных масок (обозначена переменной M), 
а также шесть параметров, задающих диапазоны значений для перебора.

Для параметра \(c\), используемого в качестве порога схожести между бинаризациями, задаются начальное значение \(c_0\), конечное значение \(c_1\) и шаг \(step_c\). Аналогично, для параметра \(\alpha\), определяющего минимальный индекс набора углов, с которого разрешается применение правила останова, задаются значения \(\alpha_0\), \(\alpha_1\) и \(step_{\alpha}\).

На основе этих значений формируются соответствующие дискретные множества параметров с равномерным шагом, которые затем используются для перебора всех возможных комбинаций в рамках эксперимента.

Следующий блок инициализирует начальное значение \(alpha\) и переменную result, в которую будет записываться результат. 

На каждый индекс \(\alpha\) в переменную result будет записано \(\frac{(c_1 - c_0)}{ step_c}\) значений. После итерации по индексам \(alpha\) в results также будут все \(\alpha_1 + 1\) значений метрик без применения правила останова в качестве контроля. 

После записи данных планируется их изобразить на графике, соответственно необходимо выбрать структуру данных для переменной result, которая позволяет наиболее эффективно выполнять ее функцию в процедуре.

В качестве структуры данных была выбрана последовательность пар массивов библиотеки NumPy, организованная в виде списка. Такая форма хранения позволяет удобно добавлять новые массивы с результатами на каждой итерации, а также напрямую передавать их в функции визуализации библиотеки Matplotlib без дополнительной обработки, поскольку формат данных уже соответствует ожидаемому.

В качестве альтернатив рассматривались двумерная матрица и словарь. Однако использование матрицы затруднено из-за различного количества углов и их значений в каждой итерации, а словарь уступает списку по скорости обработки в типичных сценариях, что делает его менее предпочтительным.

После инициализации переменных в процедуре выполняется цикл по значениям \(\alpha \in [\alpha_0, \alpha_1]\).

На каждой его итерации инициализируется начальное значение порога \(c = c_0\) и запускается вложенный цикл по значениям порога \(c \in [c_0, c_1]\).

В вложенном цикле по порогам \(c\) запускается блок рассчета среднего по набору данных значение угла и метрики при применении правила останова, обозначенного на рисунке \ref*{fig:procedurestoppingrule} цифрой 1.

В данном блоке для фиксированной пары параметров \(\alpha, c\) запускается симуляция процесса томографии под контролем реконструкции с применением правила останова на основе схожести бинаризаций.

В частности для каждого изображения в наборе данных, список который доступен в ResultData, извлекаются значения метрики M как для соседних бинаризаций, так и для эталонной маски.

В вложенном цикле по порогам \(c\) инициализируются numpy массивы для записи значений углов и метрик, после чего запускается блок расчёта среднего по набору данных значения угла и метрики при применении правила останова, обозначенного на рисунке~\ref*{fig:procedurestoppingrule} цифрой 1.

В данном блоке для фиксированной пары параметров \(\alpha, c\) осуществляется симуляция процесса томографии под контролем реконструкции с применением правила останова, основанного на схожести бинаризаций.

Для каждого изображения в наборе данных, список которых доступен через интерфейс ResultData, извлекаются значения выбранной метрики \(M\), измеряющей степень различия между бинаризациями соседних реконструкций, а также значения метрики, измеряющей отличие от эталонной маски.

Далее запускается цикл по каждому значению индекса набора углов \(i\), начиная с \(i = \alpha\). На каждой итерации производится проверка, превышает ли значение метрики между бинаризациями \(M((\hat{\image})_{i-1}, (\hat{\image})_i)\) заданный порог \(c\). Если условие выполнено, текущий индекс считается моментом останова, и дальнейший перебор для данного изображения прекращается.

После фиксации точки останова записывается соответствующий угол, на котором был выполнен останов, и значение метрики \(M\) оценивающее бинаризацию реконструкции под индексом \(i\) по сравнению с эталонной маской.

Эти значения добавляются в массивы результатов, которые впоследствии усредняются по всем изображениям.

Полученное среднее значение угла и метрики для порога \(c\) добавляются в соответствующие массивы numpy, инициализированными перед началом цикла по \(c\).

После завершения цикла по значениям \(c\), заполненные массивы numpy с значениями углов и метрик добавляются как пара tuple в исходный список result.

Данные шаги повторяются на каждой итерации внешнего цикла по \(\alpha\) и к моменту завершения цикла мы получаем заполненный список results.

После завершения цикла по значениям \(\alpha\), в блоке добавления контрольных значений (обозначен на рисунке~\ref*{fig:procedurestoppingrule} цифрой 2), к переменной result добавляется список средних значений количества углов и соответствующей метрики по эталонам без применения правила останова. Соответственно эти данные являются контролем, с которым можно сравнивать результаты, полученные при различных значениях параметров правила останова.

Далее в процедуре переменная result передается в блок вывода, который формирует графики и визуализацию результата, после чего процесс завершается.

Описанная выше процедура позволяет симулировать применение правила останова с различными параметрами и сравнить их с контрольными значениями без применения правила останова.

Далее будут рассмотрены результаты запуска данной процедуры и проведен их анализ.

\subsection{Результаты}

Реализованного конвейера экспериментов достаточно и всех его модулей достаточно чтобы провести главный эксперимент от реконструкции всех изображений по указанной в настройках стратерии, до вывода результата симуляции работы правила останова в томографии под контролем реконструкции.

Перед запуском основного эксперимента необходимо определить все нужные входные параметры.

Запуск модуля реконструкции требует определения стратегии набора углов. Такая настройка принимается модулем в виде JSON файла. Файл с настройками стратегий представлен в листинге \ref*{lst:strategysettings}

\begin{lstlisting}[language=json, caption={Файл конфигурации стратегий набора углов для конвейера экспериментов}, label={lst:strategysettings}]
[
    {
        "strategy": "binary",
        "max_angles": 2048
    }
]
\end{lstlisting}

Объект настройки стратегии находится в списке так как модуль способен выполнять реконструкцию по нескольким стратегиям набора углов.

Список необходимых ключей в самом объекте зависит от стратегии, однако для всех обязательным является ключ name. На данный момент реализована стратегия binary - удвоение углов, описанная в работе \cite{gilmanov2024applicability}.

Стратегия binary настраивается максимальным количеством углов. В рамках текущего запуска конвейера экспериментов в качестве такого значения указано 1024, соответственно \(2^{10} = 1024\), а начинаем набор углов с \(2^2 = 4\), откуда получается 9 наборов углов на каждое изображение.

Запуск реализованной в модуле визуализации процедуры экспериментов по исследованию эффективности правила останова требует определения шести аргументов, задающих границы параметров правила останова. Этими аргументами являются значения начала и конца отрезка соответствующего параметра, а также шаг в каждой итерации.

Для параметра \(\alpha\) выбран отрезок \(\alpha \in [3, 8], \alpha \in z\). Так как индексами являются целые числа, то в качестве шага \(step_{\alpha}\) выбрана единица: \(step_{\alpha} = 1\). В экспериментах из предыдущих модулей 9 наборов углов с индексами от 0 до 8.

Для параметра \(c\), определяющего пороговое значение метрики схожести бинаризаций, выбран отрезок \(c \in [0.4, 1.0]\) с шагом \(step_c = 0.005\). Таким образом, в цикле перебираются 120 различных значений параметра \(c\).

Конвейер экспериментов был запущен с описанными выше параметрами и в результате его работы были получены результаты в виде графиков и таблиц, описывающих зависимость среднего угла от среднего значения метрик. 

Таких результатов девять - по  два графика и одной таблице на каждый алгоритм сегментации.

Рассмотрим результаты экспериментов по классическому пороговому алгоритму, определенному в формуле \ref*{eq:classicthresholding}.

График зависимости среднего угла от среднего значения метрики IOU представлен на рисунке \ref*{fig:iouthreshold}.

\gostfigure{iouthreshold}{График зависимости среднего значения угла от среднего значение метрики IOU при различных параметрах правила останова для классического порогового алгоритма}{iouthreshold}

Синия линия с квадратными точками на графике является контролем - значение угла и метрики без применения правила останова.

На графике кривые соответствующие различным значениям параметра \(\alpha\) правила останова расположены выше чем контроль, особенно после 256 углов. Однако выигрыш в качестве незначительный.

В таблице \ref*{tab:thresholdingiou} преведены результаты исследования на долю точек, находящихся над кривой контроля для каждой кривой правила останова.

Хотя в абсолютных значениях выигрыш качества незначительный, доля точек над контролем у кривых правила отсанова достаточно высока. Это означает, что существует большое количество комбинаций пар параметров \((\alpha, c)\) позволяющих получить незначительный, но выигрыш в качестве по метрике IOU при применении правила останова. 


\begin{table}[H]
\centering
\caption{Доля точек кривой, на которых значение метрики IOU превышает привышает значение метрики в  контроле}
\label{tab:thresholdingiou}
\begin{tabular}{|c|c|}
\hline
\(\alpha\) & Доля точек (\%) \\
\hline
3 & 69.231 \\
4 & 76.923 \\
5 & 84.615 \\
6 & 53.846 \\
7 & 46.154 \\
\hline
\end{tabular}
\end{table}

По метрике Symmetric Boundary DICE несколько иная картина. График для этой метрики приведен в рисунке \ref*{fig:boundarydicethreshold}, а таблица с исследованием доли точек превыщающих контроль в таблице .

\gostfigure{boundarydicethreshold}{График зависимости среднего значения угла от среднего значение метрики Symmetric boundary DICE при различных параметрах правила останова для классического порогового алгоритма}{boundarydicethreshold}

В случаес с Symmetric Boundary DICE выигрыш в качестве более значимый, чем в случае с IOU, однако доля точек, превышающих контроль, значительно ниже.

Такие результаты означают что правило останова дает выигрыш в качестве, однако необходимо подбирать его параметры для получения такого эффекта.

\begin{table}[H]
\centering
\caption{Доля точек кривой, на которых значение метрики SBD превышает привышает значение метрики в  контроле}
\label{tab:thresholdingsbd}
\begin{tabular}{|c|c|}
\hline
\(\alpha\) & Доля точек (\%) \\
\hline
3 & 23.076 \\
4 & 46.153 \\
5 & 30.769 \\
6 & 15.384 \\
7 & 53.846 \\
\hline
\end{tabular}
\end{table}

Численные значение экспериментов, на базе которых были построены графики, приведены в приложении А.

Рассмотрим результаты экспериментов по применению правила останова для алгоритма Отсу. 

На рисунках \ref*{fig:iouotsu} и \ref*{fig:boundarydiceotsu} изображены графики резульатов применения правила останова. 

Из графиков можно сделать вывод, что в отличие от случая с классическим пороговым алгоритмом, выигрыш в качестве от применения правила останова значителен для обеих метрик. 

При этом, как показывают таблицы \ref*{tab:otsuiou} и \ref*{tab:otsusbd}, значения доли точек, превышающих контроль для обеих метрик также достаточно большой.
 
Такие результаты позволяют сделать вывод о высокой эффективности правила останова для алгоритма Отсу.

Выигрыш в применении правила останова в случае алгоритма Отсу наиболее позволяет исключить влияния параметров самомго алгоритма, так как у метода Отсу вручную подбираемых параметров нет.

 
\gostfigure{iouotsu}{График зависимости среднего значения угла от среднего значение метрики IOU при различных параметрах правила останова для алгоритма Отсу}{iouotsu}

\gostfigure{boundarydiceotsu}{График зависимости среднего значения угла от среднего значение метрики Symmetric boundary DICE при различных параметрах правила останова для алгоритма Отсу}{boundarydiceotsu}

\begin{table}[H]
\centering
\caption{Доля точек кривой, на которых значение метрики IOU превышает привышает значение метрики в  контроле для алгоритма Отсу}
\label{tab:otsuiou}
\begin{tabular}{|c|c|}
\hline
\(\alpha\) & Доля точек (\%) \\
\hline
3 & 46.153 \\
4 & 46.153 \\
5 & 53.845 \\
6 & 76.923 \\
7 & 76.923 \\
\hline
\end{tabular}
\end{table}

\begin{table}[H]
\centering
\caption{Доля точек кривой, на которых значение метрики SBD превышает привышает значение метрики в  контроле для алгоритма Отсу}
\label{tab:otsusbd}
\begin{tabular}{|c|c|}
\hline
\(\alpha\) & Доля точек (\%) \\
\hline
3 & 15.384 \\
4 & 30.769 \\
5 & 69.230 \\
6 & 76.923 \\
7 & 69.230 \\
\hline
\end{tabular}
\end{table}