\section{Глава 2}

\subsection{Набор данных}

В целях исследования итеративного процесса томографии под контролем реконструкции с точки зрения качества сегментации необходимы следующие компоненты:
\begin{enumerate}

    \item Набор данных, на которых проводятся эксперименты.
    \item Алгоритмы сегментации.
    \item Метрики для оценки качества сегментации.
        
\end{enumerate}   

Рассмотрим первую компоненту - входной набор данных.

Под набором данных, в рамках данной работы, подрозумевается множество из \(n\) изображений \( (\image)_1, (\image)_2, \ldots, (\image)_n \). 

В целях простоты у всех изображения набора данных одинаковый размер.

Такое упрощение позволит сосредоточиться на вопросе останова в итеративном процесссе томографии под контролем реконструкции, пропустив этап подбора параметров реконструкции отдельно для каждого изображения в наборе данных.

Основной целью работы является исследование правила останова в итеративной томографии под контролем реконструкции. Поэтому набор данных должен моделировать различные варианты соотношения между углами сканирования и угловой структурой изображений.

Большинство изображений  в наборе данных являются синтетическими и были сгенерированы с помощью языка программирования Python.

Набор данных состоит из 8 изображений, сгруппированных по 2 на основе схожести моделируемой угловой структуры.

Все изображения набора данных имеют размер \(n = (256, 512, 512)\).

Первая пара представляет собой два изображения реконструкции бетонных плит, предоставленные в работе \cite{wagner2023comparative}.

Изображение "Бетон-1" содержит характерные горизонтально ориентированные дефекты, интерпретируемые как трещины. На рисунке~\ref{fig:concrete1} приведён один из срезов: слева реконструкция, справа — эталонная бинарная маска.

\gostfigure{concrete1}{Срез 128 бинарной маски изображение "Бетон-1" }{concrete1}

Изображение "Бетон-2" представляет собой бетонную плиту с дефектом типа раковина - пустое пространство внутри плиты.

На рисунке~\ref{fig:concrete2} приведен срез изображения реконструкции "Бетон-2" и соответствующий ему срез бинарной маски.


\gostfigure{concrete2}{Срез 128 бинарной маски изображение "Бетон-2" }{concrete2}

В этих изображениях стоит отметить преобладание объекта над фоном по количеству пикселей.

Следующей парой объектов в наборе данных являются синтетические изображения трёхмерной решётки, сформированной пересекающимися семействами параллельных плоскостей. 

В одном случае решётка строго ортогональна координатным осям. Это изображение в дальнейшем обозначается как "Решётка". На рисунке~\ref{fig:grid} представлен срез её реконструкции и соответствующий срез эталонной бинарной маски.

Такой объект включён в набор данных как модель идеального случая: углы ориентации структур внутри объекта полностью совпадают с начальными углами сканирования в процессе томографии под контролем реконструкции.

\gostfigure{grid}{Срез 128 бинарной маски изображение "Решетка"}{grid}

Вторая решётка, обозначаемая как "Наклонная решётка", формируется по тому же принципу, что и предыдущая, но все её плоскости наклонены относительно координатных осей.

!!TODO добавить конкретные углы на которые конструкция наклонена!!

Соответствующие срезы реконструкции и эталона приведены на рисунке~\ref{fig:anglegrid}.

Данный объект моделирует случай, когда начальный набор углов сканирования не совпадает с ориентацией внутренних структур объекта. При этом число таких ориентаций ограничено: решётка состоит из трёх групп плоскостей, каждая из которых наклонена под фиксированным углом.

Ожидается, что при включении в процесс реконструкции проекций под соответствующими углами произойдёт резкое улучшение качества бинаризации.

\gostfigure{anglegrid}{Срез 128 бинарной маски изображения "Наклонная решётка"}{anglegrid}

\subsection{Алгоритмы сегментации в экспериментах}

С учётом необходимости многократного применения в процессе итеративной реконструкции, для экспериментов были выбраны пороговые алгоритмы, как простые, устойчивые и не требующие сложной настройки.

Основная идея пороговых алгоритмов достаточно проста. 

Пусть дано изображение \(\image\) размера \(n = (n_1, n_2, n_3)\).

Выходом алгоритма будет сегментация \(\tilde{\image}\) изображения \(\image\), совпадающая размером с исходным изображением.

Значение каждого пикселя сегментации \(\tilde{\image}\) определяется по следующей формуле:

\begin{equation}
    \tilde{\image} (i, j, k) = 
    \begin{cases}
        1, \image(i, j, k) \geq t\\
        0, \image(i, j, k) < t
    \end{cases}
\end{equation}

Соответственно в сегментации пиксель \((i, j, k)\) классифицируется как объект если значение исходного изображения в этом пикселе имеет значение больше или равное некоторому порогу \(t\), иначе этот пиксель классифицируется как фон.

В зависимости от характера порога \(t\) выделяют глобальные и локальные пороговые алгоритмы.

В глобальном случае порог не зависит от пикселя, соответственно все пиксели сравниваются с одним значением порога.

Такими алгоритмами являются классический пороговый алгоритм и алгоритм Отсу.

В классическом пороговом алгоритме порог является параметром, соответственно он требует его априорной оценки.

Алгоритм Отсу \cite{otsu1975threshold} определяет оптимальное значение порога \(t\), максимизируя межклассовую дисперсию:

\begin{equation}
    \sigma^2(t) = \omega_0(t) \omega_1(t) \left[ \mu_0(t) - \mu_1(t) \right]^2,
\end{equation}

где \(\omega_0(t)\) и \(\omega_1(t)\) — вероятности (доли) фона и объекта при пороге \(t\), а \(\mu_0(t)\) и \(\mu_1(t)\) — соответствующие средние значения интенсивности.

Алгоритм перебирает возможные значения \(t\) и выбирает то, при котором значение \(\sigma^2(t)\) максимизируется. 

Локальные пороговые алгоритмы определяют порог для каждого пикселя, соотвественно порог \(t\) становится функцией от пикселя \(t = t(i, j, k)\).

Распрастраненным локальным пороговым алгоритмом является метод Ниблэка \cite{niblack1985introduction}.

Алгоритм Ниблэка определяет порог \(t\) для пикселя \((i, j, k)\) по следующей формуле:

\begin{equation} \label{eq:niblack}
    t(i, j, k) = \mu_r(i, j, k) + k \sigma_r(i, j, k)
\end{equation}

где \(\mu_r(i, j, k)\) и \(\sigma_r(i, j, k)\) - среднее и среднеквадратичное отклонение интенсивности в окрестности \(r\) пикселя \((i, j, k)\).

В общем случае среднее и среднеквадратичное отклонение в \\ окрестности \(r\) пикселя \((i, j, k)\) рассчитываются по следующим формулам:

\begin{equation}\label{eq:niblack_mean}
    \mu_r(i, j, k) = \frac{1}{(2r + 1)^3} \sum_{x = i - r}^{i + r} \sum_{y = j - r}^{j + r} \sum_{z = k - r}^{k + r} \image(x, y, z)
\end{equation}

\begin{equation}\label{eq:niblack_std}
    \sigma_r(i, j, k) = \sqrt{\frac{1}{(2r + 1)^3} \sum_{x = i - r}^{i + r} \sum_{y = j - r}^{j + r} \sum_{z = k - r}^{k + r} (\image(x, y, z) - \mu_r(i, j, k))^2}
\end{equation}

Соответственно окном является куб с длинной сторон \(2r + 1\). 

Проблемной частью формул \ref{eq:niblack_mean} и \ref{eq:niblack_std} является ситуация, когда окно частично выходит за границы изображения.

В таких случаях применяются следующие стратегии обработки:

\begin{enumerate}
    \item \textbf{Обрезка окна по границам.} Вычисления проводятся только по той части окна, которая полностью попадает внутрь изображения.
    
    \item \textbf{Задание фиксированного значения.} За пределами изображения значения пикселей считаются равными фиксированной константе, например, нулю или среднему значению изображения.
    
    \item \textbf{Отражение по границе.} Отсутствующие значения заполняются за счёт зеркального отражения пикселей относительно соответствующей границы изображения.
\end{enumerate}

У каждой стратегии есть свои преимущества и недостатки. Её выбор зависит от характера входных изображений.

У классического алгоритма Ниблэка существует множетсо модификаций.

Такими модификациями являются, например, алгоритмы Сауволы \cite{sauvola2000adaptive} и Фансалкара \cite{phansalkar2011adaptive}.

В рамках данной работы выбран аффинный вариант алгоритма Ниблэка, описанный в работе \cite{николаев2013критерии}.

Метод аффинного Ниблэка добавляет в уравнение \ref{eq:niblack} дополнительный параметр \(beta\), который является глобальной оценкой шума.

\begin{equation}\label{eq:niblack_affine}
    t(i, j, k) = \mu_r(i, j, k) + \sigma_r(i, j, k) + \beta
\end{equation}

Такая модификация является вычислительно эффективной так как к формуле добавляется константа, однако она добавляет дополнительный параметр. 

Таким образом, итоговый набор алгоритмов, используемых в конвейере экспериментов, включает в себя:

\begin{enumerate}
    \item Классический пороговый алгоритм
    \item Алгоритм Отсу
    \item Алгоритм аффинного Ниблека
\end{enumerate}

Выбор данных алгоритмов обусловлен их высокой степенью изученности в литературе и активным применением в задачах компьютерной томографии.

Поскольку алгоритм бинаризации запускается на каждой итерации томографии под контролем реконструкции, его вычислительная эффективность напрямую влияет на общую скорость всего процесса.

В целях оценки результата работы алгоритмов сегментации необходим набор метрик, позыоляющих сформировать наиболее целостное предстовление о полученных бинарных масках.

\subsection{Метрики оценки качества сегментации}

Последним необходимым элементом экспериментального конвейера является набор метрик оценки качества сегментации.

Метрики должны быть подобраны так, чтобы в совокупности охватывать ключевые аспекты качества бинарных масок — например, точность границ и степень перекрытия с эталоном. 

Такой набор метрик должен выявлять различные типы ошибок и предоставлять более объективную оценку результата.

Хотя в области машинного обучения и компьютерного зрения существует множество метрик, не все из них применимы к задачи сегментации изображений. 

Например, метрики, ориентированные на числовые значения, текстовые последовательности или графовые структуры, не учитывают пространственные особенности изображений и потому неинформативны в контексте сегментации.

Даже среди метрик, предназначенных для работы с изображениями, нередко встречается избыточность: несколько показателей могут быть чувствительны к одним и тем же видам ошибок, и их совместное использование не даёт дополнительной информации.

В конвейер экспериментов требуется сформировать минимально необходимый набор  метрик, который позволит обнаружить распрастраненные типы ошибок.

Чтобы выбрать такой набор метрик обоснованно, необходимо провести серию экспериментов на модельных данных.

Такие эксперименты позволят изучить поведение метрик при различных типах ошибок.

Для реализации таких экспериментов необходимы следующие шаги:

\begin{enumerate}
    \item Собрать начальный набор метрик для исследования.
    \item Определить набор ошибок, которые должны обнаружить метрики.
    \item Сформировать модельные изображения для набора ошибок
\end{enumerate}

Часть метрик для начального набора были найдены в работе \cite{taha2015metrics}. Авторы данного исследования изложили распрастраненные метрики, применяемые в оценки качества сегментации изображений реконструкции в медицине.

Из таких метрик в начальный набор данных будут включены метрики DICE, Intersection Over Union (IOU), Mean Square Error (MSE).

Пусть даны эталонная бинарная маска \((\tilde{\image})_{gt}\) изображения и его сегментация \((\tilde{\image})_{seg}\).

Обозначим как \((\tilde{\image})_{gt} \cap (\tilde{\image})_{seg}\) множество пикселей, на которых значение сегментации и эталонной бинарной маски совпадают и равно 1: \((\tilde{\image})_{gt} \cap (\tilde{\image})_{seg} := |\{x \in P | (\tilde{\image})_{gt}(x) = (\tilde{\image})_{seg}(x) = 1 \}|\).

Определим также \((\tilde{\image})_{gt} \cup (\tilde{\image})_{seg}\) как множество пикселей, на которых значение сегментации или эталонной бинарной маски  равно 1: \((\tilde{\image})_{gt} \cup (\tilde{\image})_{seg} := |\{x \in P | (\tilde{\image})_{gt}(x) = 1 \lor (\tilde{\image})_{seg}(x) = 1 \}|\).

Тогда метрика DICE определяется как следующей формулой:

\begin{equation}\label{eq:DICE}
    DICE((\tilde{\image})_{seg}, (\tilde{\image})_{gt}) =
    \frac{2 \cdot \left|(\tilde{\image})_{gt} \cap (\tilde{\image})_{seg}\right|}
         {\left|(\tilde{\image})_{gt}\right| + \left|(\tilde{\image})_{seg}\right|}
\end{equation}
где \(\left|(\tilde{\image})\right|\) является мощностью множества пикселей, на которых  маска \((\tilde{\image})\) имеет значение 1: \(\left|(\tilde{\image})\right| = |\{x \in P | (\tilde{\image})(x) = 1 \}|\).

Метрика DICE принимает значения в отрезке от 0 до 1, где 0 - полное несовпадение изображений, а 1 - их полное совпадение.

Индекс DICE является одной из классических метрик перекрытия — класса метрик, которые измеряют степень пересечения двух множеств (в данном случае бинарных масок). 

Другой растрастраненной метрикой перекрытия является IOU, также известная как метрика Жаккарда. Она определяется слудующей формулой:

\begin{equation}\label{eq:IOU}
    IOU((\tilde{\image})_{seg}, (\tilde{\image})_{gt}) =
    \frac{\left|(\tilde{\image})_{gt} \cap (\tilde{\image})_{seg}\right|}
         {\left|(\tilde{\image})_{gt} \cup (\tilde{\image})_{seg}\right|}
\end{equation}

Аналогично метрики DICE, значения IOU находятся в отрезке от 0 до 1, где 0 - полное несовпадение изображений, а 1 - их полное совпадение.

Следующая метрика в начальном наборе - метрика Normalized Hausdorff - отличается от предыдущих двух по принципу сравнения.

Данная метрика основана не на площади перекрытия, а на расстоянии между множествами объекта в сегментации и в эталонной бинарной маске. 

Чтобы определить метрику Normalized Hausdorff, определим расстояние между пикселем \(x \in P\) и множеством \((\tilde{\image}) = \{x | (\tilde{\image})(x) = 1\}\).

\begin{equation}\label{eq:setDistance}
    d(x,(\tilde{\image})) = inf_{y\in (\tilde{\image})} (\sqrt{(x_0 - y_0)^2 + (x_1 - y_1)^2 + (x_2 - y_2)^2})
\end{equation}

Операясь на формулу \ref{eq:setDistance} можно определить расстояние Хаусдорфа:

\begin{equation}\label{eq:HausdorffDistance}
    HD((\tilde{\image})_{seg}, (\tilde{\image})_{gt}) = max(sup_{x \in (\tilde{\image})_{seg}}d(x, (\tilde{\image})_{gt}))
\end{equation}

Соответственно значением расстояния Хаусдорфа является максимум из минимумов расстояний от пикселя до множества объектов.

Определив все необходимые формулы можно определить метрику Normalized Hausdorff Distance:

\begin{equation}\label{eq:NormalizedHausdorffDistance}
    NHD((\tilde{\image})_{seg}, (\tilde{\image})_{gt}) = 1 - \frac{max(HD((\tilde{\image})_{seg}, (\tilde{\image})_{gt}), HD((\tilde{\image})_{gt}, (\tilde{\image})_{seg}))}{\sqrt{n_0^2 + n_1^2 + n_2^2}}
\end{equation}

Максимальное значение расстояния Хаусдорфа в пределах изображения не превышает его диагонали, поэтому длина диагонали является разумным нормализующим коэффициентом.

Чем меньше значение расстояния Хаусдорфа, тем ближе бинарные маски друг к другу.

Для удобства интерпретации нормализованное расстояние вычитается из единицы, поэтому метрика Normalized Hausdorff Distance принимает значения от 0 (полное несовпадение) до 1 (полное совпадение масок).

Последней метрикой в начальном наборе является MSE - среднеквадратичное расстояние между пикселями сравниваемых масок.

\begin{equation}\label{eq:MSE}
    MSE((\tilde{\image})_{seg}, (\tilde{\image})_{gt}) = \frac{1}{n_0 \cdot n_1 \cdot n_2} \sum_{x \in P} ((\tilde{\image})_{seg}(x) - (\tilde{\image})_{gt}(x))^2
\end{equation}

Таким ооразом, начальный набор метрик составлен из различных по принципу оценивания метрик.

Следующем шагом в формировании итогового набора метрик является составление набора эталонных изображений лоя оценки чувствительности начального набора к различным видам ошибок.

Будем рассматривать в качестве типовых ошибок рябь, соответствие границ объекта, выброс и ошибки при преобладании объекта над фоном.

Рябь будем определять как наличие фоновых пикселей, ошибочно попавших в область объекта.

Ошибка соответствия границ возникает в тех случаях, когда сегментированная область в целом совпадает с объектом, но границы объекта смещены относительно эталона. 

Следующим типом ошибок является выброс - некоторое небольшое количество пикселей ошибочно классифицированные как объект, при этом находящихся на расстоянии от области объекта.

Последей ситуацией, на которой будем проводить опыты над метриками, является преобладание объекта над фоном. 

В такой ситуации некоторые метрики могу быть менее чувствительными к ошибкам так как в пбсолютных цифрах большинство изображения является объектам. На пример это может привести к хорошей оценки нечувствительной метрикой если просто классифицировать все изображение как объект.

Модельные изображения, на которых будут проводиться эксперименты с метриками, приведены в рисунке~\ref{fig:testsmetrics}.

\gostfigure{testsmetrics}{Тестовое изображения для экспериментов по оценке чувствительности метрик; а — эталонная маска,
б — тест <<Рябь>>,
в — тест <<Соответствие границ>>,
г — тест <<Выброс>>,
д — тест <<Преобладание объекта над фоном>>,
е — эталонная маска для теста <<Преобладание объекта над фоном>>
}{testsmetrics}

Собрав начальный набор метрик, определив типы ошибок и модельные изображения, можно провести эксперименты по оценке чувствительности метрик.

Эксперименты реализованы с помощью языка программирования Python с использованием библиотек numpy для работы с изображениями как с многомерным массивом, PIL для операций чтения и записи изображений в файловой системе, а также matplotlib для визуализации результатов.

Настройки экспериментов читаются с файла формата JSON, который представляет собой массив из объектов, описывающих эксперимент. 

Каждый объект описание эксперимента содержит его название и пути к модельному и эталонному изображениям.

Программа сравнивает по каждой метрике модельные изображения с эталонными и записывает результат в таблицу, столбцы которой являются значениями метрик, а строки соответствуют экспериментам.

Заполнив таблицу с резульататами программа выводит ее визуализацию в виде табличной тепловой карты.

Получившаяся в результате экспериментов тепловая карта представлена на рисунке \ref{fig:metricsexperiment}.

\gostfigure{metricsexperiment}{Тепловая карта значений метрик для модельных изображений с различными типами ошибок. Цвет и числа отражают чувствительность метрик к отклонениям от эталона.}{metricsexperiment}

Сравнение значений метрик показывает, что IOU и DICE в целом демонстрируют схожее поведение на различных типах ошибок, однако IOU в большинстве случаев оказывается более чувствительной: она сильнее снижается при наличии отклонений от эталона. Сответственно в итоговый набор метрик имеет  смысл включить IOU и исключить DICE.

Symmetric boundary dice показывает высокую чувствительность к ошибкам соответсвия границ, ряби и ошибкам при преобладании объекта над фоном. 

Метрика MSE реагирует на все виды ошибок, однако её показания в совокупности дублируются метриками IOU и Symmetric Boundary Dice, что делает её избыточной в составе итогового набора.

Для ошибки типа выброс все метрики показывают сравнительно низкую чувствительность, однако наилучшие результаты наблюдаются у метрик IOU и Symmetric Boundary Dice.

В итоговый набор метрик включены IOU и Symmetric Boundary Dice. Первая представляет собой компактную и широко используемую метрику перекрытия, служащую ориентиром для оценки общей точности сегментации, а вторая — наиболее чувствительна к локальным погрешностям и искажениям границ.

