\section{Глава 2}

\subsection{Сегментация}

В рамках данной работы рассматривается поведение сегментации изображения реконструкции в процессе томографии под контролем реконструкции. При этом выбор алгоритмов сегментации является важным шагом для дальнейшего проведения экспериментов.

Алгоритмы сегментации,  которые будут включены в конвеер экспериментов, должны обладать следующими свойствами:

\begin{enumerate}
    \item Вычислительная эффективность
    \item Минимальное количество параметров
    \item Стабильно удовлетворительное качество сегментации
\end{enumerate}

Вычислительная эффективность является необходимым условием так как алгоритм сегментации предполагается запускать на реконструкции, полученной в ходе каждой итерации томографии под контролем реконструкции.

Сравнение алгоритмов по критерию вычислительной эффективности будет проводиться посредствам оценки их сложности относительно размера входных данных.

В процесе томографии под контролем реконструкции у алгоритма сегментации отсутствует априорные знания об объекте, а также его эталон.

В таких условиях подбор параметров является достатоно затрудненным. Данное обстоятельство и является обоснованием второго критерия - минимального количества параметров алгоритма.

Целью исследования является изучение влияния набора углов в процессе томографии под контролем реконструкции на качество сегментации. При этом требуется минимизировать колебания в качестве, обусловленные самим алгоритмом. 

В связи с этим и был  сформирован финальный критерий - при полном наборе углов и корректной настройке алгоритма качество сегментации должно быть высоким и стабильным на всём наборе данных.

В научной перидике доступно большое количество публикаций на тему алгоритмов сегментации. Исследование \cite{zhang2006advances} показало устойчивый рост количества ежегодно предложенных алгоритмов с 1995 года по 2006 года.

Систематизация и классификация доступных алгоритмов необходима для выбора наиболее подходящих для проведения экспериментов. 

Хорошей отправной точкой в изучении доступных алгоритмов сегментации является рассмотрение схем их классификации. 

Существует множество различных схем классификации алгоритмов сегментации. Например, часто используется классификация по одному признаку, такому как способ обработки изображения или степень участия человека в процессе сегментации \cite{wirjadi2007survey}.

В работе \cite{ханыков2018классификация} также предложена схема обобщённой классификации, объединяющая несколько одно-признаковых подходов в единую структуру.

В рамках проведённого литературного обзора поиск алгоритмов сегментации был проведен на основе их классификации по принципу работы.

Первым классом алгоритмов, встречающихся в литературе, является семейство пороговых алгоритмов. 

Пороговые алгоритмы бинаризации выполняют классификацию вокселей изображения на основе заданного порогового значения интенсивности. Воксели с интенсивностью ниже порога относятся к фону, тогда как воксели с интенсивностью выше или равной порогу классифицируются как принадлежащие объекту.

Существует большое количество разнообразных пороговых алгоритмов, основыные из которых изложены в работы \cite{wirjadi2007survey}. 

На пример, часть алгоритмов устанавливают порог глобавально, другие - устанавливают его индивидуально для каждого пикселя. 

Другим широко используемым подходом к сегментации является метод роста области (region growing), основанный на объединении вокселей с близкими характеристиками по мере распространения от заданных начальных точек \cite{adams1994seeded}.

Этот метод является полуавтоматическим, то есть он требует участия пользователя в процессе своей работы. В частности требуется ввести набор начальных точек.

В литературе встречается множество модификаций этого метода, включая автоматический вариант алгоритма, не требующий указания начальной точки \cite{lin2000unseeded}. 

Несмотря на такое множество модификаций, метод роста области достаточно трудно реализуем в контексте задачи сегментации под контролем реконструкции.

Полуавтоматический характер метода и ресурсоёмкость его автоматической модификации делают его непрактичным для использования в рамках данного исследования.

Задачи сегментации и кластеризации имеют достаточно схожие постановки. Вследсвии этого встречаются множетсво работ, применяющих алгоритмы кластеризации в задаче сегментации.

В частности алгоритмы K-means \cite{sarker2017segmentation} и meanshift \cite{comaniciu2002mean} встречаются в контексте задачи сегментации.

Применимость алгоритмов кластеризации к сегментации изображений реконструкции требует дополнительного анализа, поскольку реализация их вычислительно эффективных версий представляет собой нетривиальную задачу.



