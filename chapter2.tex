\section{Глава 2}

\subsection{Сегментация}

\subsubsection{Обзор алгоритмов сегментации}

В рамках данной работы рассматривается поведение сегментации изображения реконструкции в процессе томографии под контролем реконструкции. При этом выбор алгоритмов сегментации является важным шагом для дальнейшего проведения экспериментов.

С целью выбора алгоритма необходимо провести обзор доступных алгормитмов сегментации.

В научной перидике доступно большое количество публикаций на тему алгоритмов сегментации. Исследование \cite{zhang2006advances} показало устойчивый рост количества ежегодно предложенных алгоритмов с 1995 года по 2006 года.

Систематизация и классификация доступных алгоритмов необходима для выбора наиболее подходящих для проведения экспериментов. 

Хорошей отправной точкой в изучении доступных алгоритмов сегментации является рассмотрение схем их классификации. 

Существует множество различных схем классификации алгоритмов сегментации. Например, часто используется классификация по одному признаку, такому как способ обработки изображения или степень участия человека в процессе сегментации \cite{wirjadi2007survey}.

В работе \cite{ханыков2018классификация} также предложена схема обобщённой классификации, объединяющая несколько одно-признаковых подходов в единую структуру.

В рамках проведённого литературного обзора поиск алгоритмов сегментации был проведен на основе их классификации по принципу работы.

Первым классом алгоритмов, встречающихся в литературе, является семейство пороговых алгоритмов. 

Пороговые алгоритмы бинаризации выполняют классификацию вокселей изображения на основе заданного порогового значения интенсивности. Воксели с интенсивностью ниже порога относятся к фону, тогда как воксели с интенсивностью выше или равной порогу классифицируются как принадлежащие объекту.

Существует большое количество разнообразных пороговых алгоритмов, основыные из которых изложены в работы \cite{wirjadi2007survey}. 

На пример, часть алгоритмов устанавливают порог глобавально, другие - устанавливают его индивидуально для каждого пикселя. 

Другим широко используемым подходом к сегментации является метод роста области (region growing), основанный на объединении вокселей с близкими характеристиками по мере распространения от заданных начальных точек \cite{adams1994seeded}.

Этот метод является полуавтоматическим, то есть он требует участия пользователя в процессе своей работы. В частности требуется ввести набор начальных точек.

В литературе встречается множество модификаций этого метода, включая автоматический вариант алгоритма, не требующий указания начальной точки \cite{lin2000unseeded}. 

Несмотря на такое множество модификаций, метод роста области достаточно трудно реализуем в контексте задачи сегментации под контролем реконструкции.

Полуавтоматический характер метода и ресурсоёмкость его автоматической модификации делают его непрактичным для использования в рамках данного исследования.

Следующим направлением, широко представленным в литературе, являются методы кластеризации. Схожесть формальных постановок задач сегментации и кластеризации способствует применению кластеризационных алгоритмов в контексте сегментации.

В частности, алгоритмы K-means \cite{sarker2017segmentation} и mean shift \cite{comaniciu2002mean} нередко применяются при решении задач сегментации.

Применимость алгоритмов кластеризации к сегментации изображений реконструкции требует дополнительного анализа, поскольку реализация их вычислительно эффективных версий представляет собой нетривиальную задачу.

Представленные ранее классы алгоритмов во многом однородны по своей структуре и принципам работы. В то время как последующие группы объединяют существенно более разнородные методы, классифицированные по более общим признакам.

Такой группой алгоритмов являются методы на базе теории графов \cite{camilus2012review}. 

Принципы работы этих методов достаточно разнообразны. Некоторые работы применяют алгоритмы поиска разрезов в графе \cite{boykov2003computing, peng2019interactive}, другие работы адаптируют алгоритм поиска максимального пока под задачу сегментации \cite{zeng2008topology}.

Также встречаются и нестандартные подходы к сегментации в этой группе алгоритмов.

Один из таких нестандартных подходов предложен в работе \cite{felzenszwalb2004efficient}, где реализован алгоритм, концептуально близкий к Unseeded Region Growing, но основанный на представлении изображения в виде взвешенного графа.

Следующей группой являются вероятностные алгоритмы. Принцип работы методов в этой группе основан на некотором априорном предположении о распределении значений пикселей в рамках изображения.

На пример методы, изложенные в работах \cite{hu2003volumetric, ayed2006unsupervised} основаны на предположении, что значение пикселей объектов в  изображении имеет распределение вейбюля.

Последней крупной группой алгоритмов, встречающейся в литературе, это методы на основе машинного обучения.

К алгоритмам на основе машинного обучения относятся как нейросетевые методы, применяемые в общей задаче сегментации изображений \cite{lu20193d, ха2016свёрточная}, так и подходы, адаптированные под изображения реконструкции в рамках томографии \cite{milletari2016v}. 

Также встречаются менее распространённые методы, включая клеточные нейронные сети \cite{liu2011industrial} и отдельные примеры полуавтоматических алгоритмов, использующих классические модели, такие как метод опорных векторов \cite{lang2022ai, gonella2019semi}.

Некоторые выявленные в литературе методы не вписываются в описанные выше категории и базируются на оригинальных, зачастую уникальных подходах. 

Алгоритмы этого типа представлены в небольшом числе работ и не формируют обособленного направления.

Одним из таких алгоритмов является Полуавтоматический SegMo \cite{nagai2019segmo}, разработаный для сегментации изображений реконструкции индустриальной томографии.

Сам алгоритм достаточно сложный в реализации и требует активного участия пользователя, соответственно не подходит для исследования томографии под контролем реконструкции, однако авторы заявляют достаточно высокое качество сегментации на выходе.

Среди редких методов также встречаются алгоритмы на основе множеств уровня, использующие эволюцию поверхностей для выделения объектов в объёме. В работе \cite{farag20043d} такой подход применён для трёхмерной сегментации сосудистой системы на данных магнитно-резонансной ангиографии.

В рамках литературного обзора были изучены пороговые алгоритмы, методы роста области, алгоритмы на основе кластеризации, теории графов, вероятностные подходы, а также методы, использующие машинное обучение и нестандартные оригинальные принципы.

Из большого разнообразия доступных алгоритмов была выбрана группа пороговых методов.

Причина выбора пороговых методов заключается в их высокой степени изученности. В литературе представлено множество работ, посвящённых их модификации, анализу и практическому применению, обладающих высокой цитируемостью.

Применение алгоритмов других групп в контексте томографии под контролем реконструкции сопряжено с рядом ограничений. 

Многие из них являются полуавтоматическими и требуют участия пользователя, что делает их непригодными для включения в итеративный процесс реконструкции. 

Классы алгоритмов с высокой вычислительной сложностью, такие как методы кластеризации и алгоритмы на основе множеств уровня, затруднительно использовать в силу необходимости их повторного запуска на каждой итерации томографического сканирования. 

Методы, основанные на нейронных сетях, требуют большого объёма размеченных данных, которые на текущий момент недоступны для рассматриваемой задачи.

Следующим этапом после выбора класса является формирование перечня конкретных алгоритмов, из которых будет производиться отбор для последующего эксперимента.

\subsection{Перечень пороговых алгоритмов сегментации}

Основная идея пороговых алгоритмов достаточно проста. 

Пусть дано изображение \(\image\) размера \(n = (n_1, n_2, n_3)\).

Выходом алгоритма будет сегментация \(\tilde{\image}\) изображения \(\image\), совпадающая размером с исходным изображением.

Значение каждого пикселя сегментации \(\tilde{\image}\) определяется по следующей формуле:

\begin{equation}
    \tilde{\image} (i, j, k) = 
    \begin{cases}
        1, \image(i, j, k) \geq t\\
        0, \image(i, j, k) < t
    \end{cases}
\end{equation}

Соответственно в сегментации пиксель \((i, j, k)\) классифицируется как объект если значение исходного изображения в этом пикселе имеет значение больше или равное некоторому порого \(t\), иначе этот пиксель классифицируется как фон.

В зависимости от характера порога \(t\) выделяют глобальные и локальные пороговые алгоритмы.

В глобальном случае порог не зависит от пикселя, соответственно все пиксели сравниваются с одним значением порога.

Такими алгоритмами являются классический пороговый алгоритм и алгоритм Отсу.

В классическом пороговом алгоритме порог является параметром, соответственно влгоритм требует его априорной оценки.

Алгоритм Отсу \cite{otsu1975threshold} определяет оптимальное значение порога \(t\), максимизируя межклассовую дисперсию:

\begin{equation}
    \sigma^2(t) = \omega_0(t) \omega_1(t) \left[ \mu_0(t) - \mu_1(t) \right]^2,
\end{equation}

где \(\omega_0(t)\) и \(\omega_1(t)\) — вероятности (доли) фона и объекта при пороге \(t\), а \(\mu_0(t)\) и \(\mu_1(t)\) — соответствующие средние значения интенсивности.

Алгоритм перебирает возможные значения \(t\) и выбирает то, при котором значение \(\sigma^2(t)\) максимизируется. 

Локальные пороговые алгоритмы определяют порог для каждого пикселя, соотвественно порог \(t\) становится функцией от пикселя \(t = t(i, j, k)\).

Распрастраненным локальным пороговым алгоритмом является метод Ниблэка \cite{niblack1985introduction}.