\documentclass[14pt]{extarticle}  % 14 пт — как по ГОСТ

% Кодировки и язык
\usepackage[utf8]{inputenc}
\usepackage[T2A]{fontenc}
\usepackage[russian]{babel}
\usepackage{csquotes}
\usepackage{mathptmx}
\usepackage{longtable}
\usepackage{lscape}
% Изображения
\usepackage{graphicx}
\graphicspath{{figures/}}

\usepackage{float}
\usepackage{caption} 
\usepackage{listings}
\usepackage{xcolor}
\usepackage{courier}

\lstdefinelanguage{json}{
    basicstyle=\rmfamily\small,
    showstringspaces=false,
    breaklines=true,
    frame=single,
    backgroundcolor=\color{gray!10},
    literate=
     *{0}{{{\color{blue}0}}}{1}
      {1}{{{\color{blue}1}}}{1}
      {2}{{{\color{blue}2}}}{1}
      {3}{{{\color{blue}3}}}{1}
      {4}{{{\color{blue}4}}}{1}
      {5}{{{\color{blue}5}}}{1}
      {6}{{{\color{blue}6}}}{1}
      {7}{{{\color{blue}7}}}{1}
      {8}{{{\color{blue}8}}}{1}
      {9}{{{\color{blue}9}}}{1}
      {:}{{{\color{red}:}}}{1}
      {,}{{{\color{red},}}}{1}
      {"}{{{\color{black}"}}}{1}
}

\lstset{language=json}

\captionsetup[figure]{
    labelsep=space,           % Разделитель между номером и текстом — пробел
    name=Рисунок,             % Замена "Figure" на "Рисунок"
    labelfont=bf,             % Жирный номер (не обязательно, но часто требуют)
    textfont=normalfont,      % Обычный шрифт текста
    justification=centering, % Центровка подписи
    singlelinecheck=false     % Не выравнивать по центру, если одна строка
}

% ГОСТ-библиография
\usepackage[backend=biber,style=gost-numeric,sorting=none]{biblatex}
\addbibresource{references.bib}

% Стандартные математические пакеты
\usepackage{amsmath, amssymb}

% Шрифт Times New Roman
\usepackage{newtxtext,newtxmath}

% Межстрочный интервал 1.5
\usepackage{setspace}
\onehalfspacing

% Поля по ГОСТ
\usepackage[a4paper,top=2cm,bottom=2cm,left=3cm,right=2cm]{geometry}

% Абзацный отступ
\setlength{\parindent}{1.25cm}
\setlength{\parskip}{0pt}  % Без вертикального отступа между абзацами

% Выравнивание по ширине
\sloppy  % уменьшает вероятность переполнения строк
\hyphenation{конт-ролем би-на-ри-за-ции ре-кон-струк-ции}

% Команды
\newcommand\image{I^n_m}
\newcommand\gostfigure[3] {
    \begin{figure}[H]
        \centering
        \includegraphics[width=0.85\textwidth]{#1}
        \caption{#2}
        \label{fig:#3}
    \end{figure}
}

\begin{document}

\section*{Аннотация}

В работе рассматривается задача оптимизации процедуры компьютерной томографии с использованием протоколов мониторинговой реконструкции. Проведено исследование применения правила останова, основанного на анализе бинаризации промежуточных реконструкций. Для оценки качества использовались метрики IOU и Symmetric Boundary DICE, а в качестве алгоритмов бинаризации применялись методы Отсу, классического порогования и аффинного Ниблэка. Результаты экспериментов показали, что предложенное правило позволяет в среднем вдвое сократить количество необходимых углов проекций без значимой потери качества бинаризации реконструкции, что открывает перспективы для снижения дозовой нагрузки и времени сканирования.

\tableofcontents

\section{Введение}

Компьютерная томография (КТ) является одним из ключевых методов неразрушающего контроля и широко применяется как в медицинской диагностике, так и в промышленности. Основной принцип КТ заключается в восстановлении внутренней структуры объекта по множеству его проекций, полученных под различными углами.

В промышленной томографии метод используется для анализа геометрии, выявления скрытых дефектов, построения CAD-моделей и контроля соответствия изделий стандартам качества. Одной из важнейших задач при этом остаётся снижение дозовой нагрузки и времени проведения эксперимента, особенно в случаях, когда объект чувствителен к излучению или когда процесс сканирования занимает достаточно длительное время.

Современным направлением развития в этой области является применение протоколов мониторинговой реконструкции, при которых процесс сканирования и реконструкции выполняется итеративно с возможностью анализа промежуточных результатов. Такой подход делает процесс сканирования anytime-алгоритмом так как позволяет остановить сканирование при достижении достаточного качества реконструкции с уже имеющимися проекциями, потенциально уменьшая дозу излучения и время сканирования.

Задача снижения дозы является актуальной как в медицине, так и в инженерии и микроэлектронике. В медицине она возникает, например, при исследовании онкологических заболеваний, некротических поражений и других патологий в органах, чувствительных к радиации. При этом в научной литературе часто рассматриваются частные или эвристические подходы. На данный момент существует лишь одно исследование, в котором протокол мониторинговой реконструкции был применён к конкретной задаче — снижению дозовой нагрузки при обнаружении COVID-19 в лёгких на КТ-изображениях \cite{bulatov2023reducing}.

В инженерии и электронике потребность в подобных протоколах особенно выражена при контроле высокоточных и чувствительных изделий — например, в микроэлектронике, оптоэлектронике и спутниковых технологиях, где применение других методов контроля не всегда позволяет гарантировать надёжность.

Одной из базовых задач в индустриальной томографии является задача бинаризации реконструкций. Её решение позволяет отбросить артефакты, сформировать CAD-модели объектов, автоматизировать поиск дефектов и выполнить контроль соответствия изделий требованиям. Задача бинаризации имеет важное прикладное значение, и точность её решения напрямую влияет на результаты последующей обработки.

На ранних этапах развития томографии (1970–1980-е гг.) подобные задачи воспринимались как перспективные направления, поскольку внимание было сосредоточено на разработке самого метода. С развитием вычислительных средств и появлением протоколов мониторинговой реконструкции появилась возможность перехода к более гибким стратегиям, включая построение правила останова — критерия, определяющего момент завершения сбора данных, когда достигнута реконструкция удовлетворительного качества.

Следует отметить, что правила останова, выбор метрик качества и архитектура протокола могут существенно различаться в зависимости от поставленной задачи. Однако, систематическое исследование таких протоколов в рамках конкретных задач — в частности, бинаризации — остаётся недостаточно проработанным.

Актуальность указанных проблем определяет цель настоящего исследования. 

\textbf{Цель работы} — разработка и исследование протоколов мониторинговой реконструкции в рамках задачи бинаризации.

Для достижения поставленной цели были поставлены следующие задачи:

\begin{enumerate}
    \item Изучение текущего состояния научных исследований по вопросам бинарной сегментации и метрик её оценки.
    \item Выбор алгоритмов бинаризации, подходящих для практического применения в условиях мониторинговой реконструкции.
    \item Подбор и генерация набора изображений для исследования поведения бинаризации в процессе томографии под контролем реконструкции.
    \item Формирование набора метрик оценки качества бинарных сегментаций.
    \item Разработка программного комплекса, реализующего конвейер экспериментов, симулирующий процесс томографии под контролем реконструкции с применением бинаризации на каждой итерации.
    \item Формулировка правила останова.
    \item Проведение численных экспериментов по оценке эффективности сформулированного правила останова с помощью реализованного программного комплекса.
    \item Анализ полученных результатов и оценка влияния правила останова на количество необходимых проекций для получения качественной бинаризации реконструкции.
\end{enumerate}

\textbf{Объектом исследования} является процесс томографии под контролем реконструкции и процедура бинарной сегментации полученных реконструкций.

\textbf{Предметом исследования} является разработка правила останова, определяющего минимально необходимый набор углов сканирования для получения качественной бинарной реконструкции.

\section{Литературный обзор}

\subsection{Определения}

В рамках текущей работы, изображением размера \(n=(n_1, n_2, n_3)\) будем называть отображение \(I_n^m : P_3 \rightarrow G\), в котором элементы области определения \(P \subset Z^3\)  называются пикселями (вокселями), а \(n_i \in N, i = 1, 2, 3\) и \((G, +)\) - аддитивная абелева группа \cite{NikolaevPhdthesis}.

\subsection{Обзор алгоритмов сегментации}

В рамках данной работы рассматривается поведение сегментации изображения реконструкции в процессе томографии под контролем реконструкции. При этом выбор алгоритмов сегментации является важным шагом для дальнейшего проведения экспериментов.

С целью выбора алгоритма необходимо провести обзор доступных алгормитмов сегментации.

В научной перидике доступно большое количество публикаций на тему алгоритмов сегментации. Исследование \cite{zhang2006advances} показало устойчивый рост количества ежегодно предложенных алгоритмов с 1995 года по 2006 года.

Систематизация и классификация доступных алгоритмов необходима для выбора наиболее подходящих для проведения экспериментов. 

Хорошей отправной точкой в изучении доступных алгоритмов сегментации является рассмотрение схем их классификации. 

Существует множество различных схем классификации алгоритмов сегментации. Например, часто используется классификация по одному признаку, такому как способ обработки изображения или степень участия человека в процессе сегментации \cite{wirjadi2007survey}.

В работе \cite{ханыков2018классификация} также предложена схема обобщённой классификации, объединяющая несколько одно-признаковых подходов в единую структуру.

В рамках проведённого литературного обзора поиск алгоритмов сегментации был проведен на основе их классификации по принципу работы.

Первым классом алгоритмов, встречающихся в литературе, является семейство пороговых алгоритмов. 

Пороговые алгоритмы бинаризации выполняют классификацию вокселей изображения на основе заданного порогового значения интенсивности. Воксели с интенсивностью ниже порога относятся к фону, тогда как воксели с интенсивностью выше или равной порогу классифицируются как принадлежащие объекту.

Существует большое количество разнообразных пороговых алгоритмов, основыные из которых изложены в работы \cite{wirjadi2007survey}. 

На пример, часть алгоритмов устанавливают порог глобавально, другие - устанавливают его индивидуально для каждого пикселя. 

Другим широко используемым подходом к сегментации является метод роста области (region growing), основанный на объединении вокселей с близкими характеристиками по мере распространения от заданных начальных точек \cite{adams1994seeded}.

Этот метод является полуавтоматическим, то есть он требует участия пользователя в процессе своей работы. В частности требуется ввести набор начальных точек.

В литературе встречается множество модификаций этого метода, включая автоматический вариант алгоритма, не требующий указания начальной точки \cite{lin2000unseeded}. 

Несмотря на такое множество модификаций, метод роста области достаточно трудно реализуем в контексте задачи сегментации под контролем реконструкции.

Полуавтоматический характер метода и ресурсоёмкость его автоматической модификации делают его непрактичным для использования в рамках данного исследования.

Следующим направлением, широко представленным в литературе, являются методы кластеризации. Схожесть формальных постановок задач сегментации и кластеризации способствует применению кластеризационных алгоритмов в контексте сегментации.

В частности, алгоритмы K-means \cite{sarker2017segmentation} и mean shift \cite{comaniciu2002mean} нередко применяются при решении задач сегментации.

Применимость алгоритмов кластеризации к сегментации изображений реконструкции требует дополнительного анализа, поскольку реализация их вычислительно эффективных версий представляет собой нетривиальную задачу.

Представленные ранее классы алгоритмов во многом однородны по своей структуре и принципам работы. В то время как последующие группы объединяют существенно более разнородные методы, классифицированные по более общим признакам.

Такой группой алгоритмов являются методы на базе теории графов \cite{camilus2012review}. 

Принципы работы этих методов достаточно разнообразны. Некоторые работы применяют алгоритмы поиска разрезов в графе \cite{boykov2003computing, peng2019interactive}, другие работы адаптируют алгоритм поиска максимального пока под задачу сегментации \cite{zeng2008topology}.

Также встречаются и нестандартные подходы к сегментации в этой группе алгоритмов.

Один из таких нестандартных подходов предложен в работе \cite{felzenszwalb2004efficient}, где реализован алгоритм, концептуально близкий к Unseeded Region Growing, но основанный на представлении изображения в виде взвешенного графа.

Следующей группой являются вероятностные алгоритмы. Принцип работы методов в этой группе основан на некотором априорном предположении о распределении значений пикселей в рамках изображения.

На пример методы, изложенные в работах \cite{hu2003volumetric, ayed2006unsupervised} основаны на предположении, что значение пикселей объектов в  изображении имеет распределение вейбюля.

Последней крупной группой алгоритмов, встречающейся в литературе, это методы на основе машинного обучения.

К алгоритмам на основе машинного обучения относятся как нейросетевые методы, применяемые в общей задаче сегментации изображений \cite{lu20193d, ха2016свёрточная}, так и подходы, адаптированные под изображения реконструкции в рамках томографии \cite{milletari2016v}. 

Также встречаются менее распространённые методы, включая клеточные нейронные сети \cite{liu2011industrial} и отдельные примеры полуавтоматических алгоритмов, использующих классические модели, такие как метод опорных векторов \cite{lang2022ai, gonella2019semi}.

Некоторые выявленные в литературе методы не вписываются в описанные выше категории и базируются на оригинальных, зачастую уникальных подходах. 

Алгоритмы этого типа представлены в небольшом числе работ и не формируют обособленного направления.

Одним из таких алгоритмов является Полуавтоматический SegMo \cite{nagai2019segmo}, разработаный для сегментации изображений реконструкции индустриальной томографии.

Сам алгоритм достаточно сложный в реализации и требует активного участия пользователя, соответственно не подходит для исследования томографии под контролем реконструкции, однако авторы заявляют достаточно высокое качество сегментации на выходе.

Среди редких методов также встречаются алгоритмы на основе множеств уровня, использующие эволюцию поверхностей для выделения объектов в объёме. В работе \cite{farag20043d} такой подход применён для трёхмерной сегментации сосудистой системы на данных магнитно-резонансной ангиографии.

В рамках литературного обзора были изучены пороговые алгоритмы, методы роста области, алгоритмы на основе кластеризации, теории графов, вероятностные подходы, а также методы, использующие машинное обучение и нестандартные оригинальные принципы.

Из большого разнообразия доступных алгоритмов была выбрана группа пороговых методов.

Причина выбора пороговых методов заключается в их высокой степени изученности. В литературе представлено множество работ, посвящённых их модификации, анализу и практическому применению, обладающих высокой цитируемостью.

Применение алгоритмов других групп в контексте томографии под контролем реконструкции сопряжено с рядом ограничений. 

Многие из них являются полуавтоматическими и требуют участия пользователя, что делает их непригодными для включения в итеративный процесс реконструкции. 

Классы алгоритмов с высокой вычислительной сложностью, такие как методы кластеризации и алгоритмы на основе множеств уровня, затруднительно использовать в силу необходимости их повторного запуска на каждой итерации томографического сканирования. 

Методы, основанные на нейронных сетях, требуют большого объёма размеченных данных, которые на текущий момент недоступны для рассматриваемой задачи.

Следующим этапом после выбора класса является формирование перечня конкретных алгоритмов, из которых будет производиться отбор для последующего эксперимента.
\section{Глава 2}

\subsection{Сегментация}

\subsubsection{Обзор алгоритмов сегментации}

В рамках данной работы рассматривается поведение сегментации изображения реконструкции в процессе томографии под контролем реконструкции. При этом выбор алгоритмов сегментации является важным шагом для дальнейшего проведения экспериментов.

С целью выбора алгоритма необходимо провести обзор доступных алгормитмов сегментации.

В научной перидике доступно большое количество публикаций на тему алгоритмов сегментации. Исследование \cite{zhang2006advances} показало устойчивый рост количества ежегодно предложенных алгоритмов с 1995 года по 2006 года.

Систематизация и классификация доступных алгоритмов необходима для выбора наиболее подходящих для проведения экспериментов. 

Хорошей отправной точкой в изучении доступных алгоритмов сегментации является рассмотрение схем их классификации. 

Существует множество различных схем классификации алгоритмов сегментации. Например, часто используется классификация по одному признаку, такому как способ обработки изображения или степень участия человека в процессе сегментации \cite{wirjadi2007survey}.

В работе \cite{ханыков2018классификация} также предложена схема обобщённой классификации, объединяющая несколько одно-признаковых подходов в единую структуру.

В рамках проведённого литературного обзора поиск алгоритмов сегментации был проведен на основе их классификации по принципу работы.

Первым классом алгоритмов, встречающихся в литературе, является семейство пороговых алгоритмов. 

Пороговые алгоритмы бинаризации выполняют классификацию вокселей изображения на основе заданного порогового значения интенсивности. Воксели с интенсивностью ниже порога относятся к фону, тогда как воксели с интенсивностью выше или равной порогу классифицируются как принадлежащие объекту.

Существует большое количество разнообразных пороговых алгоритмов, основыные из которых изложены в работы \cite{wirjadi2007survey}. 

На пример, часть алгоритмов устанавливают порог глобавально, другие - устанавливают его индивидуально для каждого пикселя. 

Другим широко используемым подходом к сегментации является метод роста области (region growing), основанный на объединении вокселей с близкими характеристиками по мере распространения от заданных начальных точек \cite{adams1994seeded}.

Этот метод является полуавтоматическим, то есть он требует участия пользователя в процессе своей работы. В частности требуется ввести набор начальных точек.

В литературе встречается множество модификаций этого метода, включая автоматический вариант алгоритма, не требующий указания начальной точки \cite{lin2000unseeded}. 

Несмотря на такое множество модификаций, метод роста области достаточно трудно реализуем в контексте задачи сегментации под контролем реконструкции.

Полуавтоматический характер метода и ресурсоёмкость его автоматической модификации делают его непрактичным для использования в рамках данного исследования.

Следующим направлением, широко представленным в литературе, являются методы кластеризации. Схожесть формальных постановок задач сегментации и кластеризации способствует применению кластеризационных алгоритмов в контексте сегментации.

В частности, алгоритмы K-means \cite{sarker2017segmentation} и mean shift \cite{comaniciu2002mean} нередко применяются при решении задач сегментации.

Применимость алгоритмов кластеризации к сегментации изображений реконструкции требует дополнительного анализа, поскольку реализация их вычислительно эффективных версий представляет собой нетривиальную задачу.

Представленные ранее классы алгоритмов во многом однородны по своей структуре и принципам работы. В то время как последующие группы объединяют существенно более разнородные методы, классифицированные по более общим признакам.

Такой группой алгоритмов являются методы на базе теории графов \cite{camilus2012review}. 

Принципы работы этих методов достаточно разнообразны. Некоторые работы применяют алгоритмы поиска разрезов в графе \cite{boykov2003computing, peng2019interactive}, другие работы адаптируют алгоритм поиска максимального пока под задачу сегментации \cite{zeng2008topology}.

Также встречаются и нестандартные подходы к сегментации в этой группе алгоритмов.

Один из таких нестандартных подходов предложен в работе \cite{felzenszwalb2004efficient}, где реализован алгоритм, концептуально близкий к Unseeded Region Growing, но основанный на представлении изображения в виде взвешенного графа.

Следующей группой являются вероятностные алгоритмы. Принцип работы методов в этой группе основан на некотором априорном предположении о распределении значений пикселей в рамках изображения.

На пример методы, изложенные в работах \cite{hu2003volumetric, ayed2006unsupervised} основаны на предположении, что значение пикселей объектов в  изображении имеет распределение вейбюля.

Последней крупной группой алгоритмов, встречающейся в литературе, это методы на основе машинного обучения.

К алгоритмам на основе машинного обучения относятся как нейросетевые методы, применяемые в общей задаче сегментации изображений \cite{lu20193d, ха2016свёрточная}, так и подходы, адаптированные под изображения реконструкции в рамках томографии \cite{milletari2016v}. 

Также встречаются менее распространённые методы, включая клеточные нейронные сети \cite{liu2011industrial} и отдельные примеры полуавтоматических алгоритмов, использующих классические модели, такие как метод опорных векторов \cite{lang2022ai, gonella2019semi}.

Некоторые выявленные в литературе методы не вписываются в описанные выше категории и базируются на оригинальных, зачастую уникальных подходах. 

Алгоритмы этого типа представлены в небольшом числе работ и не формируют обособленного направления.

Одним из таких алгоритмов является Полуавтоматический SegMo \cite{nagai2019segmo}, разработаный для сегментации изображений реконструкции индустриальной томографии.

Сам алгоритм достаточно сложный в реализации и требует активного участия пользователя, соответственно не подходит для исследования томографии под контролем реконструкции, однако авторы заявляют достаточно высокое качество сегментации на выходе.

Среди редких методов также встречаются алгоритмы на основе множеств уровня, использующие эволюцию поверхностей для выделения объектов в объёме. В работе \cite{farag20043d} такой подход применён для трёхмерной сегментации сосудистой системы на данных магнитно-резонансной ангиографии.

В рамках литературного обзора были изучены пороговые алгоритмы, методы роста области, алгоритмы на основе кластеризации, теории графов, вероятностные подходы, а также методы, использующие машинное обучение и нестандартные оригинальные принципы.

Из большого разнообразия доступных алгоритмов была выбрана группа пороговых методов.

Причина выбора пороговых методов заключается в их высокой степени изученности. В литературе представлено множество работ, посвящённых их модификации, анализу и практическому применению, обладающих высокой цитируемостью.

Применение алгоритмов других групп в контексте томографии под контролем реконструкции сопряжено с рядом ограничений. 

Многие из них являются полуавтоматическими и требуют участия пользователя, что делает их непригодными для включения в итеративный процесс реконструкции. 

Классы алгоритмов с высокой вычислительной сложностью, такие как методы кластеризации и алгоритмы на основе множеств уровня, затруднительно использовать в силу необходимости их повторного запуска на каждой итерации томографического сканирования. 

Методы, основанные на нейронных сетях, требуют большого объёма размеченных данных, которые на текущий момент недоступны для рассматриваемой задачи.

Следующим этапом после выбора класса является формирование перечня конкретных алгоритмов, из которых будет производиться отбор для последующего эксперимента.

\subsection{Перечень пороговых алгоритмов сегментации}

Основная идея пороговых алгоритмов достаточно проста. 

Пусть дано изображение \(\image\) размера \(n = (n_1, n_2, n_3)\).

Выходом алгоритма будет сегментация \(\tilde{\image}\) изображения \(\image\), совпадающая размером с исходным изображением.

Значение каждого пикселя сегментации \(\tilde{\image}\) определяется по следующей формуле:

\begin{equation}
    \tilde{\image} (i, j, k) = 
    \begin{cases}
        1, \image(i, j, k) \geq t\\
        0, \image(i, j, k) < t
    \end{cases}
\end{equation}

Соответственно в сегментации пиксель \((i, j, k)\) классифицируется как объект если значение исходного изображения в этом пикселе имеет значение больше или равное некоторому порого \(t\), иначе этот пиксель классифицируется как фон.

В зависимости от характера порога \(t\) выделяют глобальные и локальные пороговые алгоритмы.

В глобальном случае порог не зависит от пикселя, соответственно все пиксели сравниваются с одним значением порога.

Такими алгоритмами являются классический пороговый алгоритм и алгоритм Отсу.

В классическом пороговом алгоритме порог является параметром, соответственно влгоритм требует его априорной оценки.

Алгоритм Отсу \cite{otsu1975threshold} определяет оптимальное значение порога \(t\), максимизируя межклассовую дисперсию:

\begin{equation}
    \sigma^2(t) = \omega_0(t) \omega_1(t) \left[ \mu_0(t) - \mu_1(t) \right]^2,
\end{equation}

где \(\omega_0(t)\) и \(\omega_1(t)\) — вероятности (доли) фона и объекта при пороге \(t\), а \(\mu_0(t)\) и \(\mu_1(t)\) — соответствующие средние значения интенсивности.

Алгоритм перебирает возможные значения \(t\) и выбирает то, при котором значение \(\sigma^2(t)\) максимизируется. 



\section{Глава 3}

\subsection{Правило останова}

Томография под контролем реконструкции является итеративным процессов, соответственно в каждой итерации необходимо принять решение об остановке процесса на текущей реконструкции или добавления большего количества проекций.

В рамках текущей работы изучается автоматическое принятие такогого решение заранее сформулированным правилом останова на основе анализа бинаризации.

Для изучения данного вопроса требуется сформулировать такое правило останова и рассмотреть его эффект на среднее качество бинаризации по изображениям на количество углов с помощью реализованного конвейера экспериментов.

В реальных условиях применения правила останова в процессе томографии под контролем реконструкции отсутствует доступ к эталонным данным, соответственно правило останова, основанное на объективной оценке качества, является неприменимым в данном контексте.

Однако собранный для конвейера экспериментов набор метрик позволяет всесторонне оценить схожесть двух входных изображений.

В процессе томографии под контролем реконструкции как правило в первой же итерации, с набором из 4 углов, маловероятно получить достаточно качественное изображения для остановки процесса. 

Соответственно, для формулировки осмысленного правила останова требуется сравнение двух реконструкций — полученной на текущем наборе углов и реконструкции, соответствующей предыдущему набору. 

Предполагается, что на каждой итерации доступны изображения \((\image)_{i-1}\) и \((\image)_i\), где \(i = 1, ..., n\) это индекс набора углов, а реконструкция, полученная на первом наборе, имеет индекс \(0\).

Правилом останова назовем отображение \(S : R \rightarrow \{0, 1\}\), где значение 0 означает продолжение процесса томографии под контролем реконструкции, а 1 - завершение процесса на текущей реконструкции.

Сравнивая бинаризации двух соседних реконструкций можно получить информацию о степени изменения бинаризации с добавлением большего количества проекций.

Аргументом для правила останова будет являться значение метрики \(M\) на бинаризациях \(\hat{(\image)} _{i-1}\) и \(\hat{(\image)}_i\). 

Так как объективная оценка качества бинаризации не доступно при приминении правило останова, его формулировка требует некоторых предположений и эвристики.

Первым предположением, лежащим в основе формулировки правила останова, является допущение о том, что при достаточно высокой степени схожести бинаризаций двух соседних реконструкций добавление новых проекций не приводит к значимым изменениям результата. Формально это выражается условием:
\begin{equation} \label{eq:stoppingv1}
    S(\hat{\image}_{i-1}, \hat{\image}_{i}) =
    \begin{cases}
        1, & \text{если } M((\hat{\image})_{i-1}, (\hat{\image})_i) \geq c \\
        0, & \text{иначе}
    \end{cases}
\end{equation}
где \(M\) — выбранная метрика сравнения бинарных масок, а \(c = const \in [0, 1]\) — фиксированное константное пороговое значение.

Описанное правило уже позволяет получить хороший результат в случае когда качество реконструекции с увеличением количества углов плавно улучшается.

Примерами таких изображений в наборе данных конвейера экспериментов являются объёмы "Кролик" и "Статуэтка".

В случаях, когда наблюдается резкий скачок качества при включении проекций под углами, совпадающими с основными структурными элементами исследуемого объекта, правило останова, описанное в формуле~\eqref{eq:stoppingv1}, может привести к слишком раннему завершению процесса. Это связано с тем, что до момента скачка несколько соседних реконструкций могут оказаться достаточно схожими между собой, несмотря на то, что качество ещё не достигло оптимального уровня.

Такими свойствами в наборе данных обладают изображения "Решётка" и "Наклонная решётка". В первом случае, благодаря ортогональной структуре плоскостей, приемлемое качество достигается уже при первом наборе углов. Во втором случае качество улучшается скачкообразно — при добавлении углов, совпадающих с наклонами отдельных семейств плоскостей.

Такие ситуации можно учесть в правиле останова добавив дополнительное условие на минимальное количество углов в наборе. 

Хотя в текущем виде правило останова не может сработать на самом первом наборе углов — для расчёта метрики требуется как минимум две реконструкции — этого недостаточно для устранения проблемы преждевременной остановки. Чтобы её избежать, необходимо дополнительно ограничить минимальный индекс, с которого правило начинает применяться.

Модифицируем формулу \eqref{eq:stoppingv1} - добавим в нее условие останова с определенного индекса:

\begin{equation} \label{eq:stoppingv2}
    S(\hat{\image}_{i-1}, \hat{\image}_{i}) =
    \begin{cases}
        1, & \text{если } M((\hat{\image})_{i-1}, (\hat{\image})_i) \geq c \text{ и } i \geq \alpha \\
        0, & \text{иначе}
    \end{cases}
\end{equation}
где \(\alpha = \text{const}\) — минимальный индекс, начиная с которого применяется правило останова.

Такая модификация позволяет снизить вероятность останова на локальном максимуме.

Сформулированное в формуле \eqref{eq:stoppingv2} правило останова носит предварительный характер, однако его достаточно для изучения вопроса возможности применения правил останова в томографии под контролем реконструкции на базе их бинаризаций.

Далее спроектируем эксперименты для исследования эффективности правила останова.

\subsection{Эксперименты над правилом останова и их результаты}

Модуль оценки, в режиме соседних бинаризаций, реализованного конвейера экспериментов уже содержит значения метрик по различию бинаризаций реконструкций с соседними наборами углов. Соответственно все необходимые данные для изучения эффекта правила останова уже включены в результаты конвейера экспериментов.

Эксперименты с правилом останова заключаются в переборе его параметров - минимального индекса угла \(\alpha\) и порога схожести соседних бинаризаций \(c\).

Так как данные уже доступны из результатов предыдущих модулей конвейера экспериментов, для исследования правила останова не требкется изменения текущего исходного кода - достаточно в модуле визуализации добавить новую компоненту в директорию visualizators.

Модуль визуализации обеспечивает доступ ко всем экспериментальным данным через единый интерфейс ResultData и передает его в качестве аргумента всем компонентам.

Реализация экспериментов с првилом останова предполагает симуляцию его применения в процессе томографии под контролем реконструкции.

Процедура проведения экспериментов с правилом останова описана в рисунке \ref*{fig:procedurestoppingrule}.

\gostfigure{procedurestoppingrule}{Процедура проведения экспериментов по исследованию эффективности правила останова; 1 - Блок рассчета среднего по набору данных значение угла и метрики при применении правила останова; 2 - Блок добавления контроля - значений углов и метрик без применения правила останова}{procedurestoppingrule}

Процедура принимает восемь аргументов: объект ResultData, 
имя метрики сравнения бинарных масок (обозначена переменной M), 
а также шесть параметров, задающих диапазоны значений для перебора.

Для параметра \(c\), используемого в качестве порога схожести между бинаризациями, задаются начальное значение \(c_0\), конечное значение \(c_1\) и шаг \(step_c\). Аналогично, для параметра \(\alpha\), определяющего минимальный индекс набора углов, с которого разрешается применение правила останова, задаются значения \(\alpha_0\), \(\alpha_1\) и \(step_{\alpha}\).

На основе этих значений формируются соответствующие дискретные множества параметров с равномерным шагом, которые затем используются для перебора всех возможных комбинаций в рамках эксперимента.

Следующий блок инициализирует начальное значение \(alpha\) и переменную result, в которую будет записываться результат. 

На каждый индекс \(\alpha\) в переменную result будет записано \(\frac{(c_1 - c_0)}{ step_c}\) значений. После итерации по индексам \(alpha\) в results также будут все \(\alpha_1 + 1\) значений метрик без применения правила останова в качестве контроля. 

После записи данных планируется их изобразить на графике, соответственно необходимо выбрать структуру данных для переменной result, которая позволяет наиболее эффективно выполнять ее функцию в процедуре.

В качестве структуры данных была выбрана последовательность пар массивов библиотеки NumPy, организованная в виде списка. Такая форма хранения позволяет удобно добавлять новые массивы с результатами на каждой итерации, а также напрямую передавать их в функции визуализации библиотеки Matplotlib без дополнительной обработки, поскольку формат данных уже соответствует ожидаемому.

В качестве альтернатив рассматривались двумерная матрица и словарь. Однако использование матрицы затруднено из-за различного количества углов и их значений в каждой итерации, а словарь уступает списку по скорости обработки в типичных сценариях, что делает его менее предпочтительным.

После инициализации переменных в процедуре выполняется цикл по значениям \(\alpha \in [\alpha_0, \alpha_1]\).

На каждой его итерации инициализируется начальное значение порога \(c = c_0\) и запускается вложенный цикл по значениям порога \(c \in [c_0, c_1]\).

В вложенном цикле по порогам \(c\) запускается блок рассчета среднего по набору данных значение угла и метрики при применении правила останова, обозначенного на рисунке \ref*{fig:procedurestoppingrule} цифрой 1.

В данном блоке для фиксированной пары параметров \(\alpha, c\) запускается симуляция процесса томографии под контролем реконструкции с применением правила останова на основе схожести бинаризаций.

В частности для каждого изображения в наборе данных, список который доступен в ResultData, извлекаются значения метрики M как для соседних бинаризаций, так и для эталонной маски.

В вложенном цикле по порогам \(c\) инициализируются numpy массивы для записи значений углов и метрик, после чего запускается блок расчёта среднего по набору данных значения угла и метрики при применении правила останова, обозначенного на рисунке~\ref*{fig:procedurestoppingrule} цифрой 1.

В данном блоке для фиксированной пары параметров \(\alpha, c\) осуществляется симуляция процесса томографии под контролем реконструкции с применением правила останова, основанного на схожести бинаризаций.

Для каждого изображения в наборе данных, список которых доступен через интерфейс ResultData, извлекаются значения выбранной метрики \(M\), измеряющей степень различия между бинаризациями соседних реконструкций, а также значения метрики, измеряющей отличие от эталонной маски.

Далее запускается цикл по каждому значению индекса набора углов \(i\), начиная с \(i = \alpha\). На каждой итерации производится проверка, превышает ли значение метрики между бинаризациями \(M((\hat{\image})_{i-1}, (\hat{\image})_i)\) заданный порог \(c\). Если условие выполнено, текущий индекс считается моментом останова, и дальнейший перебор для данного изображения прекращается.

После фиксации точки останова записывается соответствующий угол, на котором был выполнен останов, и значение метрики \(M\) оценивающее бинаризацию реконструкции под индексом \(i\) по сравнению с эталонной маской.

Эти значения добавляются в массивы результатов, которые впоследствии усредняются по всем изображениям.

Полученное среднее значение угла и метрики для порога \(c\) добавляются в соответствующие массивы numpy, инициализированными перед началом цикла по \(c\).

После завершения цикла по значениям \(c\), заполненные массивы numpy с значениями углов и метрик добавляются как пара tuple в исходный список result.

Данные шаги повторяются на каждой итерации внешнего цикла по \(\alpha\) и к моменту завершения цикла мы получаем заполненный список results.

После завершения цикла по значениям \(\alpha\), в блоке добавления контрольных значений (обозначен на рисунке~\ref*{fig:procedurestoppingrule} цифрой 2), к переменной result добавляется список средних значений количества углов и соответствующей метрики по эталонам без применения правила останова. Соответственно эти данные являются контролем, с которым можно сравнивать результаты, полученные при различных значениях параметров правила останова.

Далее в процедуре переменная result передается в блок вывода, который формирует графики и визуализацию результата, после чего процесс завершается.

Описанная выше процедура позволяет симулировать применение правила останова с различными параметрами и сравнить их с контрольными значениями без применения правила останова.

Далее будут рассмотрены результаты запуска данной процедуры и проведен их анализ.

\subsection{Результаты}

Реализованного конвейера экспериментов достаточно и всех его модулей достаточно чтобы провести главный эксперимент от реконструкции всех изображений по указанной в настройках стратерии, до вывода результата симуляции работы правила останова в томографии под контролем реконструкции.

Перед запуском основного эксперимента необходимо определить все нужные входные параметры.

Запуск модуля реконструкции требует определения стратегии набора углов. Такая настройка принимается модулем в виде JSON файла. Файл с настройками стратегий представлен в листинге \ref*{lst:strategysettings}

\begin{lstlisting}[language=json, caption={Файл конфигурации стратегий набора углов для конвейера экспериментов}, label={lst:strategysettings}]
[
    {
        "strategy": "binary",
        "max_angles": 2048
    }
]
\end{lstlisting}

Объект настройки стратегии находится в списке так как модуль способен выполнять реконструкцию по нескольким стратегиям набора углов.

Список необходимых ключей в самом объекте зависит от стратегии, однако для всех обязательным является ключ name. На данный момент реализована стратегия binary - удвоение углов, описанная в работе \cite{gilmanov2024applicability}.

Стратегия binary настраивается максимальным количеством углов. В рамках текущего запуска конвейера экспериментов в качестве такого значения указано 1024, соответственно \(2^{10} = 1024\), а начинаем набор углов с \(2^2 = 4\), откуда получается 9 наборов углов на каждое изображение.

Запуск реализованной в модуле визуализации процедуры экспериментов по исследованию эффективности правила останова требует определения шести аргументов, задающих границы параметров правила останова. Этими аргументами являются значения начала и конца отрезка соответствующего параметра, а также шаг в каждой итерации.

Для параметра \(\alpha\) выбран отрезок \(\alpha \in [3, 8], \alpha \in z\). Так как индексами являются целые числа, то в качестве шага \(step_{\alpha}\) выбрана единица: \(step_{\alpha} = 1\). В экспериментах из предыдущих модулей 9 наборов углов с индексами от 0 до 8.

Для параметра \(c\), определяющего пороговое значение метрики схожести бинаризаций, выбран отрезок \(c \in [0.4, 1.0]\) с шагом \(step_c = 0.005\). Таким образом, в цикле перебираются 120 различных значений параметра \(c\).

Конвейер экспериментов был запущен с описанными выше параметрами и в результате его работы были получены результаты в виде графиков и таблиц, описывающих зависимость среднего угла от среднего значения метрик. 

Таких результатов девять - по  два графика и одной таблице на каждый алгоритм сегментации.

Рассмотрим результаты экспериментов по классическому пороговому алгоритму, определенному в формуле \ref*{eq:classicthresholding}.

График зависимости среднего угла от среднего значения метрики IOU представлен на рисунке \ref*{fig:iouthreshold}.

\gostfigure{iouthreshold}{График зависимости среднего значения угла от среднего значение метрики IOU при различных параметрах правила останова для классического порогового алгоритма}{iouthreshold}

Синия линия с квадратными точками на графике является контролем - значение угла и метрики без применения правила останова.

На графике кривые соответствующие различным значениям параметра \(\alpha\) правила останова расположены выше чем контроль, особенно после 256 углов. Однако выигрыш в качестве незначительный.

В таблице \ref*{tab:thresholdingiou} преведены результаты исследования на долю точек, находящихся над кривой контроля для каждой кривой правила останова.

Хотя в абсолютных значениях выигрыш качества незначительный, доля точек над контролем у кривых правила отсанова достаточно высока. Это означает, что существует большое количество комбинаций пар параметров \((\alpha, c)\) позволяющих получить незначительный, но выигрыш в качестве по метрике IOU при применении правила останова. 


\begin{table}[H]
\centering
\caption{Доля точек кривой, на которых значение метрики IOU превышает привышает значение метрики в  контроле}
\label{tab:thresholdingiou}
\begin{tabular}{|c|c|}
\hline
\(\alpha\) & Доля точек (\%) \\
\hline
3 & 69.231 \\
4 & 76.923 \\
5 & 84.615 \\
6 & 53.846 \\
7 & 46.154 \\
\hline
\end{tabular}
\end{table}

По метрике Symmetric Boundary DICE несколько иная картина. График для этой метрики приведен в рисунке \ref*{fig:boundarydicethreshold}, а таблица с исследованием доли точек превыщающих контроль в таблице .

\gostfigure{boundarydicethreshold}{График зависимости среднего значения угла от среднего значение метрики Symmetric boundary DICE при различных параметрах правила останова для классического порогового алгоритма}{boundarydicethreshold}

В случаес с Symmetric Boundary DICE выигрыш в качестве более значимый, чем в случае с IOU, однако доля точек, превышающих контроль, значительно ниже.

Такие результаты означают что правило останова дает выигрыш в качестве, однако необходимо подбирать его параметры для получения такого эффекта.

\begin{table}[H]
\centering
\caption{Доля точек кривой, на которых значение метрики SBD превышает привышает значение метрики в  контроле}
\label{tab:thresholdingsbd}
\begin{tabular}{|c|c|}
\hline
\(\alpha\) & Доля точек (\%) \\
\hline
3 & 23.076 \\
4 & 46.153 \\
5 & 30.769 \\
6 & 15.384 \\
7 & 53.846 \\
\hline
\end{tabular}
\end{table}

Численные значение экспериментов, на базе которых были построены графики, приведены в приложении А.

Рассмотрим результаты экспериментов по применению правила останова для алгоритма Отсу. 

На рисунках \ref*{fig:iouotsu} и \ref*{fig:boundarydiceotsu} изображены графики резульатов применения правила останова. 

Из графиков можно сделать вывод, что в отличие от случая с классическим пороговым алгоритмом, выигрыш в качестве от применения правила останова значителен для обеих метрик. 

При этом, как показывают таблицы \ref*{tab:otsuiou} и \ref*{tab:otsusbd}, значения доли точек, превышающих контроль для обеих метрик также достаточно большой.
 
Такие результаты позволяют сделать вывод о высокой эффективности правила останова для алгоритма Отсу.

Выигрыш в применении правила останова в случае алгоритма Отсу наиболее позволяет исключить влияния параметров самомго алгоритма, так как у метода Отсу вручную подбираемых параметров нет.

 
\gostfigure{iouotsu}{График зависимости среднего значения угла от среднего значение метрики IOU при различных параметрах правила останова для алгоритма Отсу}{iouotsu}

\gostfigure{boundarydiceotsu}{График зависимости среднего значения угла от среднего значение метрики Symmetric boundary DICE при различных параметрах правила останова для алгоритма Отсу}{boundarydiceotsu}

\begin{table}[H]
\centering
\caption{Доля точек кривой, на которых значение метрики IOU превышает привышает значение метрики в  контроле для алгоритма Отсу}
\label{tab:otsuiou}
\begin{tabular}{|c|c|}
\hline
\(\alpha\) & Доля точек (\%) \\
\hline
3 & 46.153 \\
4 & 46.153 \\
5 & 53.845 \\
6 & 76.923 \\
7 & 76.923 \\
\hline
\end{tabular}
\end{table}

\begin{table}[H]
\centering
\caption{Доля точек кривой, на которых значение метрики SBD превышает привышает значение метрики в  контроле для алгоритма Отсу}
\label{tab:otsusbd}
\begin{tabular}{|c|c|}
\hline
\(\alpha\) & Доля точек (\%) \\
\hline
3 & 15.384 \\
4 & 30.769 \\
5 & 69.230 \\
6 & 76.923 \\
7 & 69.230 \\
\hline
\end{tabular}
\end{table}
\section{Выводы}

Разработка принципов управления процедурами компьютерной томографии и их научная обоснованность - одно из приоритетных направлений междисциплинарных исследований, находящихся на стыке современной прикладной математики, информатики, физики.  

Проведённое исследование протоколов мониторинговой реконструкции в рамках задачи бинаризации показывает, что применение правила останова позволяет повысить среднее качество бинаризации по исследуемым метрикам при меньшем числе углов проекций по сравнению с контролем, под которым в данном случае понимается остановка процесса на фиксированном заранее заданном наборе углов.~

В исследовании было сформулировано правило останова процедуры компьютерной томографии под контролем реконструкции опираясь на анализ бинаризаций. 

В работы был собран набор данных, состоящий из десяти изображений, четыри из которых являются сканированными изображениями реальных объектов, а остальные шесть синтетические.

В исследовании использовались метрики IOU и Symmetric Boundary DICE, а также алгоритмы бинаризации: Отсу, классический пороговый и аффинный Ниблэк.

В работе был спроектирован и разработан программный комплекс для конвейера экспериментов, включающий в себя модули реконструкции, бинаризации, оценки бинаризаций и модуль визуализации результатов. Реализация программного комплекса выполнена на языке Python с использованием библиотек numpy, cupy, matplotlib и astra toolbox.

Было сформировано правило останова, настраиваемое двумя параметрами - \(alpha\) и \(c\), первый из которых регулирует минимальный индекс набора углов, для начала действия правила останова, а второй - порог схожести соседних бинаризаций.

Осуществлен комплекс численных экспериментов, позволяющий установить эффективность разработанного правила.

В результате экспериментов было установлено, что применение правила останова позволяет достичь качества, сравнимого с использованием полного набора углов, в среднем за вдвое меньшее количество проекций.

Разница в качестве при этом, в среднем по всем алгоритмам, не привышает 0.02 едриницы по всем метрикам.

Наиболее значимый результат был получен с алгоритмом Отсу и метрикой IOU. Из графика \ref*{fig:iouotsu} на значении порога \(c = 0.95\) правила останова наблюдалось снижение дозы до 4 раз по сравнению с полным набором углов, при снижении качества в 0.01 единицу.

Результаты работы показывают, что применение правила останова в томографии под контролем реконструкции позволяет завершить процесс при меньшем числе углов проекций. Это, в свою очередь, способствует снижению дозы излучения и сокращению времени, необходимого для проведения КТ-исследования.

Полученные результаты открывают направления для дальнейших исследований, включая оптимизацию параметров правила останова, его проверку в практических условиях КТ-сканирования и адаптацию под различные типы данных. Перспективным также является изучение альтернативных алгоритмов бинаризации и метрик для применения в условиях anytime-реконструкции.

\printbibliography
\appendix
\renewcommand{\thesection}{\Asbuk{section}} 
\renewcommand{\thetable}{\thesection.\arabic{table}} 
\setcounter{table}{0}

\section*{ПРИЛОЖЕНИЕ А}
Результаты экспериментов для классического порогового алгоритма по метрике Symmetric Boundary DICE
\addcontentsline{toc}{section}{Приложение А. Результаты экспериментов для классического порогового алгоритма по метрикам IOU и Symmetric Boundary DICE}
\label{app:classicthresholding}


\begin{center}
\begin{longtable}{|r|r|r|r|r|r|}
\caption{Результаты экспериментов для классического порогового алгоритма: сравнение метрик IOU и Symmetric Boundary DICE} 
\label{tab:classic-thresholding} \\
\hline
\(\alpha\) & порог \(c\) & Угол (IOU) & Метрика IOU & Угол (SBD) & Метрика SBD \\
\hline
\endfirsthead

\hline
\(\alpha\) & порог \(c\) & Угол (IOU) & Метрика IOU & Угол (SBD) & Метрика SBD \\
\hline
\endhead

\hline
\endfoot

\hline
\endlastfoot
3 & 0.4 & 41 & 0.7898 & 35 & 0.6237 \\
\hline
3 & 0.45 & 41 & 0.7898 & 35 & 0.6237 \\
\hline
3 & 0.5 & 57 & 0.8449 & 35 & 0.6237 \\
\hline
3 & 0.55 & 57 & 0.8449 & 51 & 0.6444 \\
\hline
3 & 0.6 & 57 & 0.8449 & 54 & 0.6592 \\
\hline
3 & 0.65 & 105 & 0.8472 & 67 & 0.6582 \\
\hline
3 & 0.7 & 182 & 0.8531 & 89 & 0.7153 \\
\hline
3 & 0.75 & 192 & 0.8542 & 169 & 0.7148 \\
\hline
3 & 0.8 & 211 & 0.8766 & 198 & 0.717 \\
\hline
3 & 0.85 & 288 & 0.8786 & 272 & 0.7398 \\
\hline
3 & 0.9 & 288 & 0.8786 & 377 & 0.7871 \\
\hline
3 & 0.95 & 342 & 0.8855 & 416 & 0.799 \\
\hline
3 & 1.0 & 934 & 0.8984 & 934 & 0.8088 \\
\hline
4 & 0.4 & 70 & 0.8341 & 64 & 0.6998 \\
\hline
4 & 0.45 & 70 & 0.8341 & 64 & 0.6998 \\
\hline
4 & 0.5 & 83 & 0.8525 & 64 & 0.6998 \\
\hline
4 & 0.55 & 83 & 0.8525 & 76 & 0.7145 \\
\hline
4 & 0.6 & 83 & 0.8525 & 76 & 0.7145 \\
\hline
4 & 0.65 & 128 & 0.8545 & 89 & 0.7135 \\
\hline
4 & 0.7 & 204 & 0.8605 & 108 & 0.741 \\
\hline
4 & 0.75 & 211 & 0.8616 & 185 & 0.7334 \\
\hline
4 & 0.8 & 230 & 0.884 & 211 & 0.7359 \\
\hline
4 & 0.85 & 307 & 0.8859 & 281 & 0.75 \\
\hline
4 & 0.9 & 307 & 0.8859 & 384 & 0.7878 \\
\hline
4 & 0.95 & 358 & 0.8893 & 422 & 0.7997 \\
\hline
4 & 1.0 & 934 & 0.8984 & 934 & 0.8088 \\
\hline
5 & 0.4 & 128 & 0.8564 & 128 & 0.7516 \\
\hline
5 & 0.45 & 128 & 0.8564 & 128 & 0.7516 \\
\hline
5 & 0.5 & 140 & 0.8748 & 128 & 0.7516 \\
\hline
5 & 0.55 & 140 & 0.8748 & 128 & 0.7516 \\
\hline
5 & 0.6 & 140 & 0.8748 & 128 & 0.7516 \\
\hline
5 & 0.65 & 179 & 0.8772 & 140 & 0.7507 \\
\hline
5 & 0.7 & 256 & 0.8832 & 153 & 0.7643 \\
\hline
5 & 0.75 & 256 & 0.8832 & 230 & 0.7567 \\
\hline
5 & 0.8 & 268 & 0.8878 & 256 & 0.7592 \\
\hline
5 & 0.85 & 345 & 0.8897 & 320 & 0.7723 \\
\hline
5 & 0.9 & 345 & 0.8897 & 409 & 0.7975 \\
\hline
5 & 0.95 & 396 & 0.8931 & 435 & 0.8001 \\
\hline
5 & 1.0 & 934 & 0.8984 & 934 & 0.8088 \\
\hline
6 & 0.4 & 256 & 0.8829 & 256 & 0.7808 \\
\hline
6 & 0.45 & 256 & 0.8829 & 256 & 0.7808 \\
\hline
6 & 0.5 & 256 & 0.8829 & 256 & 0.7808 \\
\hline
6 & 0.55 & 256 & 0.8829 & 256 & 0.7808 \\
\hline
6 & 0.6 & 256 & 0.8829 & 256 & 0.7808 \\
\hline
6 & 0.65 & 281 & 0.8869 & 256 & 0.7808 \\
\hline
6 & 0.7 & 358 & 0.8929 & 256 & 0.7808 \\
\hline
6 & 0.75 & 358 & 0.8929 & 332 & 0.7733 \\
\hline
6 & 0.8 & 358 & 0.8929 & 358 & 0.7758 \\
\hline
6 & 0.85 & 435 & 0.8948 & 409 & 0.781 \\
\hline
6 & 0.9 & 435 & 0.8948 & 486 & 0.8048 \\
\hline
6 & 0.95 & 486 & 0.8982 & 512 & 0.8074 \\
\hline
6 & 1.0 & 1024 & 0.8984 & 1024 & 0.8087 \\
\hline
7 & 0.4 & 512 & 0.889 & 512 & 0.7869 \\
\hline
7 & 0.45 & 512 & 0.889 & 512 & 0.7869 \\
\hline
7 & 0.5 & 512 & 0.889 & 512 & 0.7869 \\
\hline
7 & 0.55 & 512 & 0.889 & 512 & 0.7869 \\
\hline
7 & 0.6 & 512 & 0.889 & 512 & 0.7869 \\
\hline
7 & 0.65 & 512 & 0.889 & 563 & 0.8068 \\
\hline
7 & 0.7 & 563 & 0.8983 & 614 & 0.8031 \\
\hline
7 & 0.75 & 563 & 0.8983 & 614 & 0.8031 \\
\hline
7 & 0.8 & 563 & 0.8983 & 614 & 0.8031 \\
\hline
7 & 0.85 & 614 & 0.8949 & 665 & 0.8083 \\
\hline
7 & 0.9 & 614 & 0.8949 & 665 & 0.8083 \\
\hline
7 & 0.95 & 665 & 0.8983 & 665 & 0.8083 \\
\hline
7 & 1.0 & 1024 & 0.8984 & 1024 & 0.8087 \\
\hline
8 & 0.4 & 1024 & 0.8984 & 1024 & 0.8087 \\
\hline
8 & 0.45 & 1024 & 0.8984 & 1024 & 0.8087 \\
\hline
8 & 0.5 & 1024 & 0.8984 & 1024 & 0.8087 \\
\hline
8 & 0.55 & 1024 & 0.8984 & 1024 & 0.8087 \\
\hline
8 & 0.6 & 1024 & 0.8984 & 1024 & 0.8087 \\
\hline
8 & 0.65 & 1024 & 0.8984 & 1024 & 0.8087 \\
\hline
8 & 0.7 & 1024 & 0.8984 & 1024 & 0.8087 \\
\hline
8 & 0.75 & 1024 & 0.8984 & 1024 & 0.8087 \\
\hline
8 & 0.8 & 1024 & 0.8984 & 1024 & 0.8087 \\
\hline
8 & 0.85 & 1024 & 0.8984 & 1024 & 0.8087 \\
\hline
8 & 0.9 & 1024 & 0.8984 & 1024 & 0.8087 \\
\hline
8 & 0.95 & 1024 & 0.8984 & 1024 & 0.8087 \\
\hline
8 & 1.0 & 1024 & 0.8984 & 1024 & 0.8087 \\
\hline

\end{longtable}
\end{center}


\section*{ПРИЛОЖЕНИЕ Б}
Результаты экспериментов для алгоритма Отсу по метрике Symmetric Boundary DICE
\addcontentsline{toc}{section}{Приложение Б. Результаты экспериментов для алгоритма Отсу по метрике Symmetric Boundary DICE}
\label{app:otsu}


\begin{center}
\begin{longtable}{|r|r|r|r|r|r|}
\caption{Результаты экспериментов для алгоритма Отсу: сравнение метрик IOU и Symmetric Boundary DICE} 
\label{tab:otsu-split} \\
\hline
\(\alpha\) & порог \(c\) & Угол (IOU) & Метрика IOU & Угол (SBD) & Метрика SBD \\
\hline
\endfirsthead

\hline
\(\alpha\) & порог \(c\) & Угол (IOU) & Метрика IOU & Угол (SBD) & Метрика SBD \\
\hline
\endhead

\hline
\endfoot

\hline
\endlastfoot
3 & 0.4 & 35 & 0.8236 & 32 & 0.7244 \\
\hline
3 & 0.45 & 35 & 0.8236 & 32 & 0.7244 \\
\hline
3 & 0.5 & 44 & 0.87 & 35 & 0.7118 \\
\hline
3 & 0.55 & 134 & 0.8909 & 35 & 0.7118 \\
\hline
3 & 0.6 & 134 & 0.8909 & 48 & 0.7401 \\
\hline
3 & 0.65 & 134 & 0.8909 & 48 & 0.7401 \\
\hline
3 & 0.7 & 134 & 0.8909 & 236 & 0.8335 \\
\hline
3 & 0.75 & 140 & 0.9108 & 246 & 0.8535 \\
\hline
3 & 0.8 & 163 & 0.9135 & 246 & 0.8535 \\
\hline
3 & 0.85 & 188 & 0.9198 & 249 & 0.8621 \\
\hline
3 & 0.9 & 201 & 0.9297 & 278 & 0.893 \\
\hline
3 & 0.95 & 281 & 0.9342 & 332 & 0.9112 \\
\hline
3 & 1.0 & 1024 & 0.9484 & 1024 & 0.9196 \\
\hline
4 & 0.4 & 64 & 0.8641 & 64 & 0.7756 \\
\hline
4 & 0.45 & 64 & 0.8641 & 64 & 0.7756 \\
\hline
4 & 0.5 & 70 & 0.8785 & 64 & 0.7756 \\
\hline
4 & 0.55 & 160 & 0.8994 & 64 & 0.7756 \\
\hline
4 & 0.6 & 160 & 0.8994 & 70 & 0.7948 \\
\hline
4 & 0.65 & 160 & 0.8994 & 70 & 0.7948 \\
\hline
4 & 0.7 & 160 & 0.8994 & 256 & 0.8606 \\
\hline
4 & 0.75 & 166 & 0.9194 & 262 & 0.8732 \\
\hline
4 & 0.8 & 185 & 0.9219 & 262 & 0.8732 \\
\hline
4 & 0.85 & 211 & 0.9282 & 262 & 0.8732 \\
\hline
4 & 0.9 & 224 & 0.9381 & 288 & 0.8945 \\
\hline
4 & 0.95 & 300 & 0.9391 & 339 & 0.9117 \\
\hline
4 & 1.0 & 1024 & 0.9484 & 1024 & 0.9196 \\
\hline
5 & 0.4 & 128 & 0.9053 & 128 & 0.8287 \\
\hline
5 & 0.45 & 128 & 0.9053 & 128 & 0.8287 \\
\hline
5 & 0.5 & 128 & 0.9053 & 128 & 0.8287 \\
\hline
5 & 0.55 & 217 & 0.9262 & 217 & 0.8802 \\
\hline
5 & 0.6 & 217 & 0.9262 & 217 & 0.8802 \\
\hline
5 & 0.65 & 217 & 0.9262 & 217 & 0.8802 \\
\hline
5 & 0.7 & 217 & 0.9262 & 307 & 0.8974 \\
\hline
5 & 0.75 & 217 & 0.9262 & 307 & 0.8974 \\
\hline
5 & 0.8 & 230 & 0.9263 & 307 & 0.8974 \\
\hline
5 & 0.85 & 256 & 0.9326 & 307 & 0.8974 \\
\hline
5 & 0.9 & 268 & 0.9425 & 320 & 0.9042 \\
\hline
5 & 0.95 & 345 & 0.9435 & 358 & 0.9121 \\
\hline
5 & 1.0 & 1024 & 0.9484 & 1024 & 0.9196 \\
\hline
6 & 0.4 & 256 & 0.9334 & 256 & 0.8785 \\
\hline
6 & 0.45 & 281 & 0.9083 & 256 & 0.8785 \\
\hline
6 & 0.5 & 332 & 0.9388 & 281 & 0.8972 \\
\hline
6 & 0.55 & 332 & 0.9388 & 409 & 0.9165 \\
\hline
6 & 0.6 & 332 & 0.9388 & 409 & 0.9165 \\
\hline
6 & 0.65 & 332 & 0.9388 & 409 & 0.9165 \\
\hline
6 & 0.7 & 332 & 0.9388 & 409 & 0.9165 \\
\hline
6 & 0.75 & 332 & 0.9388 & 409 & 0.9165 \\
\hline
6 & 0.8 & 332 & 0.9388 & 409 & 0.9165 \\
\hline
6 & 0.85 & 358 & 0.9451 & 409 & 0.9165 \\
\hline
6 & 0.9 & 358 & 0.9451 & 409 & 0.9165 \\
\hline
6 & 0.95 & 435 & 0.9461 & 435 & 0.919 \\
\hline
6 & 1.0 & 1024 & 0.9484 & 1024 & 0.9196 \\
\hline
7 & 0.4 & 512 & 0.9122 & 512 & 0.8692 \\
\hline
7 & 0.45 & 512 & 0.9122 & 512 & 0.8692 \\
\hline
7 & 0.5 & 563 & 0.9427 & 512 & 0.8692 \\
\hline
7 & 0.55 & 563 & 0.9427 & 614 & 0.9196 \\
\hline
7 & 0.6 & 563 & 0.9427 & 614 & 0.9196 \\
\hline
7 & 0.65 & 563 & 0.9427 & 614 & 0.9196 \\
\hline
7 & 0.7 & 563 & 0.9427 & 614 & 0.9196 \\
\hline
7 & 0.75 & 563 & 0.9427 & 614 & 0.9196 \\
\hline
7 & 0.8 & 563 & 0.9427 & 614 & 0.9196 \\
\hline
7 & 0.85 & 563 & 0.9427 & 614 & 0.9196 \\
\hline
7 & 0.9 & 563 & 0.9427 & 614 & 0.9196 \\
\hline
7 & 0.95 & 614 & 0.9467 & 614 & 0.9196 \\
\hline
7 & 1.0 & 1024 & 0.9484 & 1024 & 0.9196 \\
\hline
8 & 0.4 & 1024 & 0.9484 & 1024 & 0.9196 \\
\hline
8 & 0.45 & 1024 & 0.9484 & 1024 & 0.9196 \\
\hline
8 & 0.5 & 1024 & 0.9484 & 1024 & 0.9196 \\
\hline
8 & 0.55 & 1024 & 0.9484 & 1024 & 0.9196 \\
\hline
8 & 0.6 & 1024 & 0.9484 & 1024 & 0.9196 \\
\hline
8 & 0.65 & 1024 & 0.9484 & 1024 & 0.9196 \\
\hline
8 & 0.7 & 1024 & 0.9484 & 1024 & 0.9196 \\
\hline
8 & 0.75 & 1024 & 0.9484 & 1024 & 0.9196 \\
\hline
8 & 0.8 & 1024 & 0.9484 & 1024 & 0.9196 \\
\hline
8 & 0.85 & 1024 & 0.9484 & 1024 & 0.9196 \\
\hline
8 & 0.9 & 1024 & 0.9484 & 1024 & 0.9196 \\
\hline
8 & 0.95 & 1024 & 0.9484 & 1024 & 0.9196 \\
\hline
8 & 1.0 & 1024 & 0.9484 & 1024 & 0.9196 \\
\hline

\end{longtable}
\end{center}


\section*{ПРИЛОЖЕНИЕ В}
Результаты экспериментов для алгоритма аффинного Ниблэка по метрике Symmetric Boundary DICE
\addcontentsline{toc}{section}{Приложение В. Результаты экспериментов для алгоритма аффинного Ниблэка по метрике Symmetric Boundary DICE}
\label{app:niblack}


\begin{center}
\begin{longtable}{|r|r|r|r|r|r|}
\caption{Результаты экспериментов для аффинного алгоритма Ниблэка: сравнение метрик IOU и Symmetric Boundary DICE} 
\label{tab:affine-niblack} \\
\hline
\(\alpha\) & порог \(c\) & Угол (IOU) & Метрика IOU & Угол (SBD) & Метрика SBD \\
\hline
\endfirsthead

\hline
\(\alpha\) & порог \(c\) & Угол (IOU) & Метрика IOU & Угол (SBD) & Метрика SBD \\
\hline
\endhead

\hline
\endfoot

\hline
\endlastfoot
3 & 0.4 & 32 & 0.5666 & 38 & 0.44 \\
\hline
3 & 0.45 & 32 & 0.5666 & 73 & 0.489 \\
\hline
3 & 0.5 & 35 & 0.5741 & 112 & 0.5278 \\
\hline
3 & 0.55 & 54 & 0.5985 & 115 & 0.5262 \\
\hline
3 & 0.6 & 67 & 0.6095 & 121 & 0.543 \\
\hline
3 & 0.65 & 76 & 0.6384 & 166 & 0.5872 \\
\hline
3 & 0.7 & 92 & 0.6527 & 198 & 0.62 \\
\hline
3 & 0.75 & 124 & 0.6688 & 326 & 0.6317 \\
\hline
3 & 0.8 & 227 & 0.6786 & 451 & 0.6528 \\
\hline
3 & 0.85 & 300 & 0.6891 & 560 & 0.665 \\
\hline
3 & 0.9 & 390 & 0.7041 & 598 & 0.6783 \\
\hline
3 & 0.95 & 630 & 0.7158 & 784 & 0.6904 \\
\hline
3 & 1.0 & 1024 & 0.7222 & 1024 & 0.6937 \\
\hline
4 & 0.4 & 64 & 0.6136 & 64 & 0.4814 \\
\hline
4 & 0.45 & 64 & 0.6136 & 96 & 0.5289 \\
\hline
4 & 0.5 & 64 & 0.6136 & 134 & 0.5677 \\
\hline
4 & 0.55 & 76 & 0.6322 & 134 & 0.5677 \\
\hline
4 & 0.6 & 89 & 0.6431 & 134 & 0.5677 \\
\hline
4 & 0.65 & 96 & 0.6516 & 179 & 0.6119 \\
\hline
4 & 0.7 & 204 & 0.6643 & 211 & 0.6447 \\
\hline
4 & 0.75 & 236 & 0.6804 & 332 & 0.6453 \\
\hline
4 & 0.8 & 256 & 0.6811 & 454 & 0.6528 \\
\hline
4 & 0.85 & 307 & 0.6895 & 563 & 0.6651 \\
\hline
4 & 0.9 & 396 & 0.7045 & 601 & 0.6783 \\
\hline
4 & 0.95 & 633 & 0.7158 & 787 & 0.6904 \\
\hline
4 & 1.0 & 1024 & 0.7222 & 1024 & 0.6937 \\
\hline
5 & 0.4 & 128 & 0.6614 & 128 & 0.5302 \\
\hline
5 & 0.45 & 128 & 0.6614 & 140 & 0.5457 \\
\hline
5 & 0.5 & 128 & 0.6614 & 179 & 0.5845 \\
\hline
5 & 0.55 & 128 & 0.6614 & 179 & 0.5845 \\
\hline
5 & 0.6 & 140 & 0.6723 & 217 & 0.6296 \\
\hline
5 & 0.65 & 140 & 0.6723 & 217 & 0.6296 \\
\hline
5 & 0.7 & 243 & 0.683 & 243 & 0.6546 \\
\hline
5 & 0.75 & 268 & 0.6942 & 358 & 0.6556 \\
\hline
5 & 0.8 & 281 & 0.6944 & 473 & 0.6628 \\
\hline
5 & 0.85 & 332 & 0.7029 & 576 & 0.6699 \\
\hline
5 & 0.9 & 409 & 0.7072 & 614 & 0.6832 \\
\hline
5 & 0.95 & 640 & 0.7158 & 793 & 0.6904 \\
\hline
5 & 1.0 & 1024 & 0.7222 & 1024 & 0.6937 \\
\hline
6 & 0.4 & 256 & 0.7016 & 281 & 0.6414 \\
\hline
6 & 0.45 & 256 & 0.7016 & 281 & 0.6414 \\
\hline
6 & 0.5 & 256 & 0.7016 & 307 & 0.6627 \\
\hline
6 & 0.55 & 256 & 0.7016 & 307 & 0.6627 \\
\hline
6 & 0.6 & 256 & 0.7016 & 307 & 0.6627 \\
\hline
6 & 0.65 & 256 & 0.7016 & 307 & 0.6627 \\
\hline
6 & 0.7 & 332 & 0.7001 & 307 & 0.6627 \\
\hline
6 & 0.75 & 358 & 0.7113 & 486 & 0.6702 \\
\hline
6 & 0.8 & 358 & 0.7113 & 512 & 0.6708 \\
\hline
6 & 0.85 & 384 & 0.7108 & 614 & 0.6779 \\
\hline
6 & 0.9 & 460 & 0.7152 & 640 & 0.6858 \\
\hline
6 & 0.95 & 665 & 0.7183 & 819 & 0.693 \\
\hline
6 & 1.0 & 1024 & 0.7222 & 1024 & 0.6937 \\
\hline
7 & 0.4 & 512 & 0.7208 & 512 & 0.6833 \\
\hline
7 & 0.45 & 512 & 0.7208 & 512 & 0.6833 \\
\hline
7 & 0.5 & 512 & 0.7208 & 512 & 0.6833 \\
\hline
7 & 0.55 & 512 & 0.7208 & 512 & 0.6833 \\
\hline
7 & 0.6 & 512 & 0.7208 & 512 & 0.6833 \\
\hline
7 & 0.65 & 512 & 0.7208 & 512 & 0.6833 \\
\hline
7 & 0.7 & 563 & 0.7194 & 563 & 0.6866 \\
\hline
7 & 0.75 & 563 & 0.7194 & 614 & 0.6799 \\
\hline
7 & 0.8 & 563 & 0.7194 & 614 & 0.6799 \\
\hline
7 & 0.85 & 563 & 0.7194 & 716 & 0.687 \\
\hline
7 & 0.9 & 614 & 0.7184 & 716 & 0.687 \\
\hline
7 & 0.95 & 768 & 0.7213 & 870 & 0.6931 \\
\hline
7 & 1.0 & 1024 & 0.7222 & 1024 & 0.6937 \\
\hline
8 & 0.4 & 1024 & 0.7222 & 1024 & 0.6937 \\
\hline
8 & 0.45 & 1024 & 0.7222 & 1024 & 0.6937 \\
\hline
8 & 0.5 & 1024 & 0.7222 & 1024 & 0.6937 \\
\hline
8 & 0.55 & 1024 & 0.7222 & 1024 & 0.6937 \\
\hline
8 & 0.6 & 1024 & 0.7222 & 1024 & 0.6937 \\
\hline
8 & 0.65 & 1024 & 0.7222 & 1024 & 0.6937 \\
\hline
8 & 0.7 & 1024 & 0.7222 & 1024 & 0.6937 \\
\hline
8 & 0.75 & 1024 & 0.7222 & 1024 & 0.6937 \\
\hline
8 & 0.8 & 1024 & 0.7222 & 1024 & 0.6937 \\
\hline
8 & 0.85 & 1024 & 0.7222 & 1024 & 0.6937 \\
\hline
8 & 0.9 & 1024 & 0.7222 & 1024 & 0.6937 \\
\hline
8 & 0.95 & 1024 & 0.7222 & 1024 & 0.6937 \\
\hline
8 & 1.0 & 1024 & 0.7222 & 1024 & 0.6937 \\
\hline

\end{longtable}
\end{center}

\end{document}