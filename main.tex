\documentclass[14pt]{extarticle}  % 14 пт — как по ГОСТ

% Кодировки и язык
\usepackage[utf8]{inputenc}
\usepackage[T2A]{fontenc}
\usepackage[russian]{babel}
\usepackage{csquotes}
\usepackage{mathptmx}
\usepackage{longtable}
\usepackage{lscape}
% Изображения
\usepackage{graphicx}
\graphicspath{{figures/}}

\usepackage{float}
\usepackage{caption} 
\usepackage{listings}
\usepackage{xcolor}
\usepackage{courier}

\lstdefinelanguage{json}{
    basicstyle=\rmfamily\small,
    showstringspaces=false,
    breaklines=true,
    frame=single,
    backgroundcolor=\color{gray!10},
    literate=
     *{0}{{{\color{blue}0}}}{1}
      {1}{{{\color{blue}1}}}{1}
      {2}{{{\color{blue}2}}}{1}
      {3}{{{\color{blue}3}}}{1}
      {4}{{{\color{blue}4}}}{1}
      {5}{{{\color{blue}5}}}{1}
      {6}{{{\color{blue}6}}}{1}
      {7}{{{\color{blue}7}}}{1}
      {8}{{{\color{blue}8}}}{1}
      {9}{{{\color{blue}9}}}{1}
      {:}{{{\color{red}:}}}{1}
      {,}{{{\color{red},}}}{1}
      {"}{{{\color{black}"}}}{1}
}

\lstset{language=json}

\captionsetup[figure]{
    labelsep=space,           % Разделитель между номером и текстом — пробел
    name=Рисунок,             % Замена "Figure" на "Рисунок"
    labelfont=bf,             % Жирный номер (не обязательно, но часто требуют)
    textfont=normalfont,      % Обычный шрифт текста
    justification=centering, % Центровка подписи
    singlelinecheck=false     % Не выравнивать по центру, если одна строка
}

% ГОСТ-библиография
\usepackage[backend=biber,style=gost-numeric,sorting=none]{biblatex}
\addbibresource{references.bib}

% Стандартные математические пакеты
\usepackage{amsmath, amssymb}

% Шрифт Times New Roman
\usepackage{newtxtext,newtxmath}

% Межстрочный интервал 1.5
\usepackage{setspace}
\onehalfspacing

% Поля по ГОСТ
\usepackage[a4paper,top=2cm,bottom=2cm,left=3cm,right=2cm]{geometry}

% Абзацный отступ
\setlength{\parindent}{1.25cm}
\setlength{\parskip}{0pt}  % Без вертикального отступа между абзацами

% Выравнивание по ширине
\sloppy  % уменьшает вероятность переполнения строк
\hyphenation{конт-ролем би-на-ри-за-ции ре-кон-струк-ции}

% Команды
\newcommand\image{I^n_m}
\newcommand\gostfigure[3] {
    \begin{figure}[H]
        \centering
        \includegraphics[width=0.85\textwidth]{#1}
        \caption{#2}
        \label{fig:#3}
    \end{figure}
}

\begin{document}

\tableofcontents

\section{Введение}
Компьютерная томография (КТ) является одним из ключевых методов неразрушающего контроля и широко применяется как в медицинской диагностике, так и в промышленности. Её основной принцип заключается в восстановлении внутренней структуры объекта по множеству его проекций, полученных под различными углами.

В промышленной томографии КТ используется для анализа геометрии, выявления скрытых дефектов, построения CAD-моделей и контроля соответствия изделий стандартам качества. 

При этом одной из важнейших задач остаётся снижение дозовой нагрузки и времени эксперимента, особенно в случаях, когда объект чувствителен к излучению или необходима высокая скорость сканирования.

Одним из современных направлений развития является применение протоколов мониторинговой реконструкции, при которых процесс сканирования и реконструкции выполняется итеративно, с возможностью анализа промежуточных результатов. Такой подход призван минимизировать дозу излучения и сократить время эксперимента путем предоставления возможности остановки процесса сканирования если достигнута достаточное качество реконтсрукции с имеющимися проекциями.

Отметим, крайнюю прикладную актуальность рассматриваемой задачи, как в сфере медицины, так и инженерии, и электроники. В медицине данная проблема возникает сравнительно часто при исследовании поврежденных тканей. В частности, при КТ и рентгенографии опухолей, различной этиологии (в особенности носящих агрессивный характер), при исследовании некротических поражений, при дифференциальной диагностике осложнений и новообразований в высокочувствительных к радиации органах. В научной периодике, сравнительно часто встречаются различные эвристические и не апробированные подходы по решению данной задачи. В настоящее время существует всего одно исследование, где было выработано решение не общей постановки задачи в рамках абстрактного улучшения качества, а решена конкретная проблема - определены и обоснованы характеристики процедуры компьютерной томографии для снижения дозы радиации при детектировании COVID-19 в легких \cite{bulatov2023reducing}. 

Для инженерии и электротехники потребность в развитии протоколов мониторинговой реконструкции в компьютерной томографии прослеживается на этапе контроля качества производства оборудования и высокотехнологичных компонентов, где излишнее радиационное воздействие может оказать негативный эффект (и при этом использование иных методов неразрушающего контроля не позволит гарантировать бездефектность исследуемого объекта). Например, подобные задачи возникают при контроле микроэлектроники, элементов и узлов спутников, электронных микроскопов, высокоточной оптики. 

Задача бинаризации является одной из типовых задач в индустриальной томографии, и ее решение позволяет отбросить артефакты реконструкции. Это в свою очередь делает возможным построение меш и CAD модели исследуемых объектов, выполнение контроля производства и автоматического обнаружения дефектов.

На этапе разработки метода компьютерной томографии в 1970 – 1980 -гг., подобные идеи являлись заделом на будущее. В то время основные задачи концентрировались в части развития самого метода томографии. Теперь, с развитием компьютерной техники и программного обеспечения, и, в частности, с возникновением протоколов мониторинговой реконструкции появилась возможность разработки правила останова, для определения минимально необходимого набора углов для получения удовлетворительной бинарной сегментации реконструкций. 

При этом, следует отметить, что метрики, правила останова и реализация протокола могут существенно отличаться в зависимости от решаемой задачи. 

Вместе с тем, получение результатов по различным объектам и последующее их обобщение могут привести к разработке универсальных и общих зависимостей, что существенно повысит управляемость и качество процедуры томографии под контролем реконструкции. 

Актуальность представленных проблем, стала основой выработки цели данного исследования.
Цель данной работы - Разработка и исследование протоколов мониторинговой реконструкции в рамках задачи бинаризации.
Для достижения цели был сформирован следующий комплекс задач:
\begin{enumerate}
    \item Изучить текущую научную периодику по проблеме бинарной сегментации и метриках оценки их качества.
    \item Выбрать алгоритмы бинаризации, подходящие для практического применения в anytime процессе мониторинговой реконструкции.
    \item Подобрать и сгенерировать исходные данные для исследования поведения бинаризаций реконструкций в томографии под контролем реконструкции.
    \item Сформировать набор метрик для оценки качества бинарных сегментаций реконструкций.
    \item Разработать программный комплекс конвейера экспериментов симулирующих процесс томографии под контролем реконструкции.
    \item Сформулировать правило останова
    \item С помощью реализованного программного комплекса провести численные эксперименты по исследованию эффективности правила останова.
    \item Обработать полученные результаты и установить эффект правила останова на количество необходимых проекций для получения качественной бинаризации реконструкции
\end{enumerate}
Предмет работы: разработка правила останова для определения минимального необходимого набора углов для получения качественной бинаризации реконструкции.
Объект работы: процесс томографии под контролем реконструкции и процедура бинарной сегментации полученных реконструкций.

\section{Литературный обзор}

В рамках текущей работы, изображением размера \(n=(n_1, n_2, n_3)\) будем называть отображение \(I_n^m : P_3 \rightarrow G\), в котором элементы области определения \(P \subset Z^3\)  называются пикселями, а \(n_i \in N, i = 1, 2, 3\) и \((G, +)\) - аддитивная абелева группа \cite{NikolaevPhdthesis}.
\section{Глава 2}

\subsection{Набор данных}

В целях исследования итеративного процесса томографии под контролем реконструкции с точки зрения качества сегментации необходимы следующие компоненты:
\begin{enumerate}

    \item Набор данных, на которых проводятся эксперименты.
    \item Алгоритмы сегментации.
    \item Метрики для оценки качества сегментации.
        
\end{enumerate}   

Рассмотрим первую компоненту - входной набор данных.

Под набором данных, в рамках данной работы, подрозумевается множество из \(n\) изображений \( (\image)_1, (\image)_2, \ldots, (\image)_n \). 

В целях простоты у всех изображения набора данных одинаковый размер.

Такое упрощение позволит сосредоточиться на вопросе останова в итеративном процесссе томографии под контролем реконструкции, пропустив этап подбора параметров реконструкции отдельно для каждого изображения в наборе данных.

Основной целью работы является исследование правила останова в итеративной томографии под контролем реконструкции. Поэтому набор данных должен моделировать различные варианты соотношения между углами сканирования и угловой структурой изображений.

Большинство изображений  в наборе данных являются синтетическими и были сгенерированы с помощью языка программирования Python.

Набор данных состоит из 8 изображений, сгруппированных по 2 на основе схожести моделируемой угловой структуры.

Все изображения набора данных имеют размер \(n = (256, 512, 512)\).

Первая пара представляет собой два изображения реконструкции бетонных плит, предоставленные в работе \cite{wagner2023comparative}.

Изображение "Бетон-1" содержит характерные горизонтально ориентированные дефекты, интерпретируемые как трещины. На рисунке~\ref{fig:concrete1} приведён один из срезов: слева реконструкция, справа — эталонная бинарная маска.

\gostfigure{concrete1}{Срез 128 бинарной маски изображение "Бетон-1" }{concrete1}

Изображение "Бетон-2" представляет собой бетонную плиту с дефектом типа раковина - пустое пространство внутри плиты.

На рисунке~\ref{fig:concrete2} приведен срез изображения реконструкции "Бетон-2" и соответствующий ему срез бинарной маски.


\gostfigure{concrete2}{Срез 128 бинарной маски изображение "Бетон-2" }{concrete2}

В этих изображениях стоит отметить преобладание объекта над фоном по количеству пикселей.

Следующей парой объектов в наборе данных являются синтетические изображения трёхмерной решётки, сформированной пересекающимися семействами параллельных плоскостей. 

В одном случае решётка строго ортогональна координатным осям. Это изображение в дальнейшем обозначается как "Решётка". На рисунке~\ref{fig:grid} представлен срез её реконструкции и соответствующий срез эталонной бинарной маски.

Такой объект включён в набор данных как модель идеального случая: углы ориентации структур внутри объекта полностью совпадают с начальными углами сканирования в процессе томографии под контролем реконструкции.

\gostfigure{grid}{Срез 128 бинарной маски изображение "Решетка"}{grid}

Вторая решётка, обозначаемая как "Наклонная решётка", формируется по тому же принципу, что и предыдущая, но все её плоскости наклонены относительно координатных осей.

!!TODO добавить конкретные углы на которые конструкция наклонена!!

Соответствующие срезы реконструкции и эталона приведены на рисунке~\ref{fig:anglegrid}.

Данный объект моделирует случай, когда начальный набор углов сканирования не совпадает с ориентацией внутренних структур объекта. При этом число таких ориентаций ограничено: решётка состоит из трёх групп плоскостей, каждая из которых наклонена под фиксированным углом.

Ожидается, что при включении в процесс реконструкции проекций под соответствующими углами произойдёт резкое улучшение качества бинаризации.

\gostfigure{anglegrid}{Срез 128 бинарной маски изображения "Наклонная решётка"}{anglegrid}

\subsection{Алгоритмы сегментации в экспериментах}

С учётом необходимости многократного применения в процессе итеративной реконструкции, для экспериментов были выбраны пороговые алгоритмы, как простые, устойчивые и не требующие сложной настройки.

Основная идея пороговых алгоритмов достаточно проста. 

Пусть дано изображение \(\image\) размера \(n = (n_1, n_2, n_3)\).

Выходом алгоритма будет сегментация \(\tilde{\image}\) изображения \(\image\), совпадающая размером с исходным изображением.

Значение каждого пикселя сегментации \(\tilde{\image}\) определяется по следующей формуле:

\begin{equation}
    \tilde{\image} (i, j, k) = 
    \begin{cases}
        1, \image(i, j, k) \geq t\\
        0, \image(i, j, k) < t
    \end{cases}
\end{equation}

Соответственно в сегментации пиксель \((i, j, k)\) классифицируется как объект если значение исходного изображения в этом пикселе имеет значение больше или равное некоторому порогу \(t\), иначе этот пиксель классифицируется как фон.

В зависимости от характера порога \(t\) выделяют глобальные и локальные пороговые алгоритмы.

В глобальном случае порог не зависит от пикселя, соответственно все пиксели сравниваются с одним значением порога.

Такими алгоритмами являются классический пороговый алгоритм и алгоритм Отсу.

В классическом пороговом алгоритме порог является параметром, соответственно он требует его априорной оценки.

Алгоритм Отсу \cite{otsu1975threshold} определяет оптимальное значение порога \(t\), максимизируя межклассовую дисперсию:

\begin{equation}
    \sigma^2(t) = \omega_0(t) \omega_1(t) \left[ \mu_0(t) - \mu_1(t) \right]^2,
\end{equation}

где \(\omega_0(t)\) и \(\omega_1(t)\) — вероятности (доли) фона и объекта при пороге \(t\), а \(\mu_0(t)\) и \(\mu_1(t)\) — соответствующие средние значения интенсивности.

Алгоритм перебирает возможные значения \(t\) и выбирает то, при котором значение \(\sigma^2(t)\) максимизируется. 

Локальные пороговые алгоритмы определяют порог для каждого пикселя, соотвественно порог \(t\) становится функцией от пикселя \(t = t(i, j, k)\).

Распрастраненным локальным пороговым алгоритмом является метод Ниблэка \cite{niblack1985introduction}.

Алгоритм Ниблэка определяет порог \(t\) для пикселя \((i, j, k)\) по следующей формуле:

\begin{equation} \label{eq:niblack}
    t(i, j, k) = \mu_r(i, j, k) + k \sigma_r(i, j, k)
\end{equation}

где \(\mu_r(i, j, k)\) и \(\sigma_r(i, j, k)\) - среднее и среднеквадратичное отклонение интенсивности в окрестности \(r\) пикселя \((i, j, k)\).

В общем случае среднее и среднеквадратичное отклонение в \\ окрестности \(r\) пикселя \((i, j, k)\) рассчитываются по следующим формулам:

\begin{equation}\label{eq:niblack_mean}
    \mu_r(i, j, k) = \frac{1}{(2r + 1)^3} \sum_{x = i - r}^{i + r} \sum_{y = j - r}^{j + r} \sum_{z = k - r}^{k + r} \image(x, y, z)
\end{equation}

\begin{equation}\label{eq:niblack_std}
    \sigma_r(i, j, k) = \sqrt{\frac{1}{(2r + 1)^3} \sum_{x = i - r}^{i + r} \sum_{y = j - r}^{j + r} \sum_{z = k - r}^{k + r} (\image(x, y, z) - \mu_r(i, j, k))^2}
\end{equation}

Соответственно окном является куб с длинной сторон \(2r + 1\). 

Проблемной частью формул \ref{eq:niblack_mean} и \ref{eq:niblack_std} является ситуация, когда окно частично выходит за границы изображения.

В таких случаях применяются следующие стратегии обработки:

\begin{enumerate}
    \item \textbf{Обрезка окна по границам.} Вычисления проводятся только по той части окна, которая полностью попадает внутрь изображения.
    
    \item \textbf{Задание фиксированного значения.} За пределами изображения значения пикселей считаются равными фиксированной константе, например, нулю или среднему значению изображения.
    
    \item \textbf{Отражение по границе.} Отсутствующие значения заполняются за счёт зеркального отражения пикселей относительно соответствующей границы изображения.
\end{enumerate}

У каждой стратегии есть свои преимущества и недостатки. Её выбор зависит от характера входных изображений.

У классического алгоритма Ниблэка существует множетсо модификаций.

Такими модификациями являются, например, алгоритмы Сауволы \ref{sauvola2000adaptive} и Фансалкара \ref{phansalkar2011adaptive}.

В рамках данной работы выбрана аффинный вариант алгоритма Ниблэка, описанный в работе \ref{николаев2013критерии}.

Суть 


\section{Глава 3}

\subsection{Правило останова}

Томография под контролем реконструкции является итеративным процессов, соответственно в каждой итерации необходимо принять решение об остановке процесса на текущей реконструкции или добавления большего количества проекций.

В рамках текущей работы изучается автоматическое принятие такогого решение заранее сформулированным правилом останова на основе анализа бинаризации.

Для изучения данного вопроса требуется сформулировать такое правило останова и рассмотреть его эффект на среднее качество бинаризации по изображениям на количество углов с помощью реализованного конвейера экспериментов.

В реальных условиях применения правила останова в процессе томографии под контролем реконструкции отсутствует доступ к эталонным данным, соответственно правило останова, основанное на объективной оценке качества, является неприменимым в данном контексте.

Однако собранный для конвейера экспериментов набор метрик позволяет всесторонне оценить схожесть двух входных изображений.

В процессе томографии под контролем реконструкции как правило в первой же итерации, с набором из 4 углов, маловероятно получить достаточно качественное изображения для остановки процесса. 

Соответственно, для формулировки осмысленного правила останова требуется сравнение двух реконструкций — полученной на текущем наборе углов и реконструкции, соответствующей предыдущему набору. 

Предполагается, что на каждой итерации доступны изображения \((\image)_{i-1}\) и \((\image)_i\), где \(i = 1, ..., n\) это индекс набора углов, а реконструкция, полученная на первом наборе, имеет индекс \(0\).

Правилом останова назовем отображение \(S : R \rightarrow \{0, 1\}\), где значение 0 означает продолжение процесса томографии под контролем реконструкции, а 1 - завершение процесса на текущей реконструкции.

Сравнивая бинаризации двух соседних реконструкций можно получить информацию о степени изменения бинаризации с добавлением большего количества проекций.

Аргументом для правила останова будет являться значение метрики \(M\) на бинаризациях \(\hat{(\image)} _{i-1}\) и \(\hat{(\image)}_i\). 

Так как объективная оценка качества бинаризации не доступно при приминении правило останова, его формулировка требует некоторых предположений и эвристики.

Первым предположением, лежащим в основе формулировки правила останова, является допущение о том, что при достаточно высокой степени схожести бинаризаций двух соседних реконструкций добавление новых проекций не приводит к значимым изменениям результата. Формально это выражается условием:
\begin{equation} \label{eq:stoppingv1}
    S(\hat{\image}_{i-1}, \hat{\image}_{i}) =
    \begin{cases}
        1, & \text{если } M((\hat{\image})_{i-1}, (\hat{\image})_i) \geq c \\
        0, & \text{иначе}
    \end{cases}
\end{equation}
где \(M\) — выбранная метрика сравнения бинарных масок, а \(c = const \in [0, 1]\) — фиксированное константное пороговое значение.

Описанное правило уже позволяет получить хороший результат в случае когда качество реконструекции с увеличением количества углов плавно улучшается.

Примерами таких изображений в наборе данных конвейера экспериментов являются объёмы "Кролик" и "Статуэтка".

В случаях, когда наблюдается резкий скачок качества при включении проекций под углами, совпадающими с основными структурными элементами исследуемого объекта, правило останова, описанное в формуле~\eqref{eq:stoppingv1}, может привести к слишком раннему завершению процесса. Это связано с тем, что до момента скачка несколько соседних реконструкций могут оказаться достаточно схожими между собой, несмотря на то, что качество ещё не достигло оптимального уровня.

Такими свойствами в наборе данных обладают изображения "Решётка" и "Наклонная решётка". В первом случае, благодаря ортогональной структуре плоскостей, приемлемое качество достигается уже при первом наборе углов. Во втором случае качество улучшается скачкообразно — при добавлении углов, совпадающих с наклонами отдельных семейств плоскостей.

Такие ситуации можно учесть в правиле останова добавив дополнительное условие на минимальное количество углов в наборе. 

Хотя в текущем виде правило останова не может сработать на самом первом наборе углов — для расчёта метрики требуется как минимум две реконструкции — этого недостаточно для устранения проблемы преждевременной остановки. Чтобы её избежать, необходимо дополнительно ограничить минимальный индекс, с которого правило начинает применяться.

Модифицируем формулу \eqref{eq:stoppingv1} - добавим в нее условие останова с определенного индекса:

\begin{equation} \label{eq:stoppingv2}
    S(\hat{\image}_{i-1}, \hat{\image}_{i}) =
    \begin{cases}
        1, & \text{если } M((\hat{\image})_{i-1}, (\hat{\image})_i) \geq c \text{ и } i \geq \alpha \\
        0, & \text{иначе}
    \end{cases}
\end{equation}
где \(\alpha = \text{const}\) — минимальный индекс, начиная с которого применяется правило останова.

Такая модификация позволяет снизить вероятность останова на локальном максимуме.

Сформулированное в формуле \eqref{eq:stoppingv2} правило останова носит предварительный характер, однако его достаточно для изучения вопроса возможности применения правил останова в томографии под контролем реконструкции на базе их бинаризаций.

Далее спроектируем эксперименты для исследования эффективности правила останова.

\subsection{Эксперименты над правилом останова и их результаты}

Модуль оценки, в режиме соседних бинаризаций, реализованного конвейера экспериментов уже содержит значения метрик по различию бинаризаций реконструкций с соседними наборами углов. Соответственно все необходимые данные для изучения эффекта правила останова уже включены в результаты конвейера экспериментов.

Эксперименты с правилом останова заключаются в переборе его параметров - минимального индекса угла \(\alpha\) и порога схожести соседних бинаризаций \(c\).

Так как данные уже доступны из результатов предыдущих модулей конвейера экспериментов, для исследования правила останова не требкется изменения текущего исходного кода - достаточно в модуле визуализации добавить новую компоненту в директорию visualizators.

Модуль визуализации обеспечивает доступ ко всем экспериментальным данным через единый интерфейс ResultData и передает его в качестве аргумента всем компонентам.

Реализация экспериментов с првилом останова предполагает симуляцию его применения в процессе томографии под контролем реконструкции.

Процедура проведения экспериментов с правилом останова описана в рисунке \ref*{fig:procedurestoppingrule}.

\gostfigure{procedurestoppingrule}{Процедура проведения экспериментов по исследованию эффективности правила останова; 1 - Блок рассчета среднего по набору данных значение угла и метрики при применении правила останова; 2 - Блок добавления контроля - значений углов и метрик без применения правила останова}{procedurestoppingrule}

Процедура принимает восемь аргументов: объект ResultData, 
имя метрики сравнения бинарных масок (обозначена переменной M), 
а также шесть параметров, задающих диапазоны значений для перебора.

Для параметра \(c\), используемого в качестве порога схожести между бинаризациями, задаются начальное значение \(c_0\), конечное значение \(c_1\) и шаг \(step_c\). Аналогично, для параметра \(\alpha\), определяющего минимальный индекс набора углов, с которого разрешается применение правила останова, задаются значения \(\alpha_0\), \(\alpha_1\) и \(step_{\alpha}\).

На основе этих значений формируются соответствующие дискретные множества параметров с равномерным шагом, которые затем используются для перебора всех возможных комбинаций в рамках эксперимента.

Следующий блок инициализирует начальное значение \(alpha\) и переменную result, в которую будет записываться результат. 

На каждый индекс \(\alpha\) в переменную result будет записано \(\frac{(c_1 - c_0)}{ step_c}\) значений. После итерации по индексам \(alpha\) в results также будут все \(\alpha_1 + 1\) значений метрик без применения правила останова в качестве контроля. 

После записи данных планируется их изобразить на графике, соответственно необходимо выбрать структуру данных для переменной result, которая позволяет наиболее эффективно выполнять ее функцию в процедуре.

В качестве структуры данных была выбрана последовательность пар массивов библиотеки NumPy, организованная в виде списка. Такая форма хранения позволяет удобно добавлять новые массивы с результатами на каждой итерации, а также напрямую передавать их в функции визуализации библиотеки Matplotlib без дополнительной обработки, поскольку формат данных уже соответствует ожидаемому.

В качестве альтернатив рассматривались двумерная матрица и словарь. Однако использование матрицы затруднено из-за различного количества углов и их значений в каждой итерации, а словарь уступает списку по скорости обработки в типичных сценариях, что делает его менее предпочтительным.

После инициализации переменных в процедуре выполняется цикл по значениям \(\alpha \in [\alpha_0, \alpha_1]\).

На каждой его итерации инициализируется начальное значение порога \(c = c_0\) и запускается вложенный цикл по значениям порога \(c \in [c_0, c_1]\).

В вложенном цикле по порогам \(c\) запускается блок рассчета среднего по набору данных значение угла и метрики при применении правила останова, обозначенного на рисунке \ref*{fig:procedurestoppingrule} цифрой 1.

В данном блоке для фиксированной пары параметров \(\alpha, c\) запускается симуляция процесса томографии под контролем реконструкции с применением правила останова на основе схожести бинаризаций.

В частности для каждого изображения в наборе данных, список который доступен в ResultData, извлекаются значения метрики M как для соседних бинаризаций, так и для эталонной маски.

В вложенном цикле по порогам \(c\) инициализируются numpy массивы для записи значений углов и метрик, после чего запускается блок расчёта среднего по набору данных значения угла и метрики при применении правила останова, обозначенного на рисунке~\ref*{fig:procedurestoppingrule} цифрой 1.

В данном блоке для фиксированной пары параметров \(\alpha, c\) осуществляется симуляция процесса томографии под контролем реконструкции с применением правила останова, основанного на схожести бинаризаций.

Для каждого изображения в наборе данных, список которых доступен через интерфейс ResultData, извлекаются значения выбранной метрики \(M\), измеряющей степень различия между бинаризациями соседних реконструкций, а также значения метрики, измеряющей отличие от эталонной маски.

Далее запускается цикл по каждому значению индекса набора углов \(i\), начиная с \(i = \alpha\). На каждой итерации производится проверка, превышает ли значение метрики между бинаризациями \(M((\hat{\image})_{i-1}, (\hat{\image})_i)\) заданный порог \(c\). Если условие выполнено, текущий индекс считается моментом останова, и дальнейший перебор для данного изображения прекращается.

После фиксации точки останова записывается соответствующий угол, на котором был выполнен останов, и значение метрики \(M\) оценивающее бинаризацию реконструкции под индексом \(i\) по сравнению с эталонной маской.

Эти значения добавляются в массивы результатов, которые впоследствии усредняются по всем изображениям.

Полученное среднее значение угла и метрики для порога \(c\) добавляются в соответствующие массивы numpy, инициализированными перед началом цикла по \(c\).

После завершения цикла по значениям \(c\), заполненные массивы numpy с значениями углов и метрик добавляются как пара tuple в исходный список result.

Данные шаги повторяются на каждой итерации внешнего цикла по \(\alpha\) и к моменту завершения цикла мы получаем заполненный список results.

После завершения цикла по значениям \(\alpha\), в блоке добавления контрольных значений (обозначен на рисунке~\ref*{fig:procedurestoppingrule} цифрой 2), к переменной result добавляется список средних значений количества углов и соответствующей метрики по эталонам без применения правила останова. Соответственно эти данные являются контролем, с которым можно сравнивать результаты, полученные при различных значениях параметров правила останова.

Далее в процедуре переменная result передается в блок вывода, который формирует графики и визуализацию результата, после чего процесс завершается.

Описанная выше процедура позволяет симулировать применение правила останова с различными параметрами и сравнить их с контрольными значениями без применения правила останова.

Далее будут рассмотрены результаты запуска данной процедуры и проведен их анализ.

\subsection{Результаты}

Реализованного конвейера экспериментов достаточно и всех его модулей достаточно чтобы провести главный эксперимент от реконструкции всех изображений по указанной в настройках стратерии, до вывода результата симуляции работы правила останова в томографии под контролем реконструкции.

Перед запуском основного эксперимента необходимо определить все нужные входные параметры.

Запуск модуля реконструкции требует определения стратегии набора углов. Такая настройка принимается модулем в виде JSON файла. Файл с настройками стратегий представлен в листинге \ref*{lst:strategysettings}

\begin{lstlisting}[language=json, caption={Файл конфигурации стратегий набора углов для конвейера экспериментов}, label={lst:strategysettings}]
[
    {
        "strategy": "binary",
        "max_angles": 2048
    }
]
\end{lstlisting}

Объект настройки стратегии находится в списке так как модуль способен выполнять реконструкцию по нескольким стратегиям набора углов.

Список необходимых ключей в самом объекте зависит от стратегии, однако для всех обязательным является ключ name. На данный момент реализована стратегия binary - удвоение углов, описанная в работе \cite{gilmanov2024applicability}.

Стратегия binary настраивается максимальным количеством углов. В рамках текущего запуска конвейера экспериментов в качестве такого значения указано 1024, соответственно \(2^{10} = 1024\), а начинаем набор углов с \(2^2 = 4\), откуда получается 9 наборов углов на каждое изображение.

Запуск реализованной в модуле визуализации процедуры экспериментов по исследованию эффективности правила останова требует определения шести аргументов, задающих границы параметров правила останова. Этими аргументами являются значения начала и конца отрезка соответствующего параметра, а также шаг в каждой итерации.

Для параметра \(\alpha\) выбран отрезок \(\alpha \in [3, 8], \alpha \in z\). Так как индексами являются целые числа, то в качестве шага \(step_{\alpha}\) выбрана единица: \(step_{\alpha} = 1\). В экспериментах из предыдущих модулей 9 наборов углов с индексами от 0 до 8.

Для параметра \(c\), определяющего пороговое значение метрики схожести бинаризаций, выбран отрезок \(c \in [0.4, 1.0]\) с шагом \(step_c = 0.005\). Таким образом, в цикле перебираются 120 различных значений параметра \(c\).

Конвейер экспериментов был запущен с описанными выше параметрами и в результате его работы были получены результаты в виде графиков и таблиц, описывающих зависимость среднего угла от среднего значения метрик. 

Таких результатов девять - по  два графика и одной таблице на каждый алгоритм сегментации.

Рассмотрим результаты экспериментов по классическому пороговому алгоритму, определенному в формуле \ref*{eq:classicthresholding}.

График зависимости среднего угла от среднего значения метрики IOU представлен на рисунке \ref*{fig:iouthreshold}.

\gostfigure{iouthreshold}{График зависимости среднего значения угла от среднего значение метрики IOU при различных параметрах правила останова для классического порогового алгоритма}{iouthreshold}

Синия линия с квадратными точками на графике является контролем - значение угла и метрики без применения правила останова.

На графике кривые соответствующие различным значениям параметра \(\alpha\) правила останова расположены выше чем контроль, особенно после 256 углов. Однако выигрыш в качестве незначительный.

В таблице \ref*{tab:thresholdingiou} преведены результаты исследования на долю точек, находящихся над кривой контроля для каждой кривой правила останова.

Хотя в абсолютных значениях выигрыш качества незначительный, доля точек над контролем у кривых правила отсанова достаточно высока. Это означает, что существует большое количество комбинаций пар параметров \((\alpha, c)\) позволяющих получить незначительный, но выигрыш в качестве по метрике IOU при применении правила останова. 


\begin{table}[H]
\centering
\caption{Доля точек кривой, на которых значение метрики IOU превышает привышает значение метрики в  контроле}
\label{tab:thresholdingiou}
\begin{tabular}{|c|c|}
\hline
\(\alpha\) & Доля точек (\%) \\
\hline
3 & 69.231 \\
4 & 76.923 \\
5 & 84.615 \\
6 & 53.846 \\
7 & 46.154 \\
\hline
\end{tabular}
\end{table}

По метрике Symmetric Boundary DICE несколько иная картина. График для этой метрики приведен в рисунке \ref*{fig:boundarydicethreshold}, а таблица с исследованием доли точек превыщающих контроль в таблице .

\gostfigure{boundarydicethreshold}{График зависимости среднего значения угла от среднего значение метрики Symmetric boundary DICE при различных параметрах правила останова для классического порогового алгоритма}{boundarydicethreshold}

В случаес с Symmetric Boundary DICE выигрыш в качестве более значимый, чем в случае с IOU, однако доля точек, превышающих контроль, значительно ниже.

Такие результаты означают что правило останова дает выигрыш в качестве, однако необходимо подбирать его параметры для получения такого эффекта.

\begin{table}[H]
\centering
\caption{Доля точек кривой, на которых значение метрики SBD превышает привышает значение метрики в  контроле}
\label{tab:thresholdingsbd}
\begin{tabular}{|c|c|}
\hline
\(\alpha\) & Доля точек (\%) \\
\hline
3 & 23.076 \\
4 & 46.153 \\
5 & 30.769 \\
6 & 15.384 \\
7 & 53.846 \\
\hline
\end{tabular}
\end{table}

Численные значение экспериментов, на базе которых были построены графики, приведены в приложении А.

Рассмотрим результаты экспериментов по применению правила останова для алгоритма Отсу. 

На рисунках \ref*{fig:iouotsu} и \ref*{fig:boundarydiceotsu} изображены графики резульатов применения правила останова. 

Из графиков можно сделать вывод, что в отличие от случая с классическим пороговым алгоритмом, выигрыш в качестве от применения правила останова значителен для обеих метрик. 

При этом, как показывают таблицы \ref*{tab:otsuiou} и \ref*{tab:otsusbd}, значения доли точек, превышающих контроль для обеих метрик также достаточно большой.
 
Такие результаты позволяют сделать вывод о высокой эффективности правила останова для алгоритма Отсу.

Выигрыш в применении правила останова в случае алгоритма Отсу наиболее позволяет исключить влияния параметров самомго алгоритма, так как у метода Отсу вручную подбираемых параметров нет.

 
\gostfigure{iouotsu}{График зависимости среднего значения угла от среднего значение метрики IOU при различных параметрах правила останова для алгоритма Отсу}{iouotsu}

\gostfigure{boundarydiceotsu}{График зависимости среднего значения угла от среднего значение метрики Symmetric boundary DICE при различных параметрах правила останова для алгоритма Отсу}{boundarydiceotsu}

\begin{table}[H]
\centering
\caption{Доля точек кривой, на которых значение метрики IOU превышает привышает значение метрики в  контроле для алгоритма Отсу}
\label{tab:otsuiou}
\begin{tabular}{|c|c|}
\hline
\(\alpha\) & Доля точек (\%) \\
\hline
3 & 46.153 \\
4 & 46.153 \\
5 & 53.845 \\
6 & 76.923 \\
7 & 76.923 \\
\hline
\end{tabular}
\end{table}

\begin{table}[H]
\centering
\caption{Доля точек кривой, на которых значение метрики SBD превышает привышает значение метрики в  контроле для алгоритма Отсу}
\label{tab:otsusbd}
\begin{tabular}{|c|c|}
\hline
\(\alpha\) & Доля точек (\%) \\
\hline
3 & 15.384 \\
4 & 30.769 \\
5 & 69.230 \\
6 & 76.923 \\
7 & 69.230 \\
\hline
\end{tabular}
\end{table}
\section{Выводы}

Разработка принципов управления процедурами компьютерной томографии и их научная обоснованность - одно из приоритетных направлений междисциплинарных исследований, находящихся на стыке современной прикладной математики, информатики, физики.  

Проведённое исследование протоколов мониторинговой реконструкции в рамках задачи бинаризации показывает, что применение правила останова позволяет повысить среднее качество бинаризации по исследуемым метрикам при меньшем числе углов проекций по сравнению с контролем, под которым в данном случае понимается остановка процесса на фиксированном заранее заданном наборе углов.~

В исследовании было сформулировано правило останова процедуры компьютерной томографии под контролем реконструкции опираясь на анализ бинаризаций. 

Был проведен обзор корпуса литературы по алгоритмам бинаризации и метрикам оценки их качества. 

На основе проведенного обзора были выбраны алгоритмы сегментации, применимые в anytime условиях мониторинговой реконструкции. В частноти, были выбраны алгоритмы Отсу, аффинного Ниблэка и классический пороговый метод.

В работы был собран набор данных, состоящий из десяти изображений, четыри из которых являются сканированными изображениями реальных объектов, а остальные шесть синтетические.

Изображения, основанные на сканированых изображениях объектов включают два изображения реконструкции бетонной плиты и две трехмерные модели объектов, полученные с помощью сканнера - фигурка кролика и тайская статуэтка.

Синтетические изображения были сгенерированы с разнообразной геометрией, чтобы симулировать различные ситуации в мониторинговой реконструкции. 

Эти синтетические изображения разбиты на три пары. 

Первая такая пара - ортогональная и наклонная сетки, симулирующие изображения, которые требуют небольшое количество углов для получения достаточного уровня детализации в реконструкции.

Следующей парой являются гладкие объекты - трехмерная гауссиана и набор эллипсов. В силу гладкости этих объектов ожидается, что добавление большего количества углов проекций будет плавно повышать качество реконструкций.

Последней парой синететических объектов являются полигоны. Эти изображения состоят из двух пересекающихся пирамид и одного шестиугольника. Предполагается, что в силу наличия прямых линий и острых углов, при включении определенного набора проекций должны быть дискретные скачки в качестве.
Количество углов, полностью покрывающих все геометрические особенности этих изображений, значительно больше, чем в случае с сетками.

Далее был сформирован набор метрик на основе результатов численных экспериментов на модельных изображениях, симулирующих различные виды ошибок. Такими метриками, чувствительными ко всем видам ошибок, определенным в эксперименте, является IOU и Symmetric Boundary DICE.

В работе был спроектирован и разработан программный комплекс для конвейера экспериментов, включающий в себя модули реконструкции, бинаризации, оценки бинаризаций и модуль визуализации результатов. Реализация программного комплекса выполнена на языке Python с использованием библиотек numpy, cupy, matplotlib и astra toolbox.

Далее было сформировано правило останова, настраиваемое двумя параметрами - \(alpha\) и \(c\), первый из которых регулирует минимальный индекс набора углов, для начала действия правила останова, а второй - порог схожести соседних бинаризаций.

На основе правила останова была разработана методика для определения его эффективности. Осуществлен комплекс численных экспериментов, позволяющий установить эффективность разработанного правила.

На основе программного комплекса и разработанной методики запущены эксперменты и получены результаты, указывающие на значительный выигрыш в среднем качестве бинаризаций по обеим метрикам по сравнению с контролем для алгоритмов Отсу и классического порогового. Применение правила останова для алгоритма аффинного Ниблэка показало незначительный, хотя и положительный, результат. Очевидно это связано с спецификой самого алгоритма в большей степени, чем с правилом останова.

Среди наиболее важных результатов работы необходимо выделить следующие:
\begin{enumerate}
    \item Предложено правило останова на основе анализа качества бинаризаций реконструкций.
    \item Сформирована методологическая и программная база для проведения экспериментов по исследованию эффективности правила останова.
    \item Проведены численные эксперименты по оценке эффективности правила останова.
    \item Полученны положительные результаты при применении предложенного правила останова.
    \item Установлено, что использование бинаризации является допустимым и перспективным подходом к автоматическому определению точки останова в мониторинговой реконструкции.
\end{enumerate}

Результаты работы показывают, что применение правила останова в томографии под контролем реконструкции позволяет завершить процесс при меньшем числе углов проекций. Это, в свою очередь, способствует снижению дозы излучения и сокращению времени, необходимого для проведения КТ-исследования.

Полученные также результаты создают предпосылки для проведения ряда последующих исследований. В частности, представляет интерес изучение выбора параметров правила останова, которые можно применить для различных типов входных данных. Кроме того, важным направлением является экспериментальная проверка предложенного правила останова в практических условиях на КТ-сканировании. Перспективным также является исследование других быстрых алгоритмов бинаризации и метрик, пригодных для применения в условиях anytime-реконструкции. В дальнейшем значимым представляется проведение работы по модификации правила останова с целью построения более универсального подхода.

\printbibliography
\appendix
\renewcommand{\thesection}{\Asbuk{section}} 
\renewcommand{\thetable}{\thesection.\arabic{table}} 
\setcounter{table}{0}

\section*{ПРИЛОЖЕНИЕ А}
Результаты экспериментов для классического порогового алгоритма по метрике Symmetric Boundary DICE
\addcontentsline{toc}{section}{Приложение А. Результаты экспериментов для классического порогового алгоритма по метрикам IOU и Symmetric Boundary DICE}
\label{app:thresholdingBoundaryDice}


\begin{center}
\begin{longtable}{|r|r|r|r|r|r|}
\caption{Результаты экспериментов для классического порогового алгоритма: сравнение метрик IOU и Symmetric Boundary DICE} 
\label{tab:classic-thresholding} \\
\hline
\(\alpha\) & порог \(c\) & Угол (IOU) & Метрика IOU & Угол (SBD) & Метрика SBD \\
\hline
\endfirsthead

\hline
\(\alpha\) & порог \(c\) & Угол (IOU) & Метрика IOU & Угол (SBD) & Метрика SBD \\
\hline
\endhead

\hline
\endfoot

\hline
\endlastfoot
3 & 0.4 & 41 & 0.7898 & 35 & 0.6237 \\
\hline
3 & 0.45 & 41 & 0.7898 & 35 & 0.6237 \\
\hline
3 & 0.5 & 57 & 0.8449 & 35 & 0.6237 \\
\hline
3 & 0.55 & 57 & 0.8449 & 51 & 0.6444 \\
\hline
3 & 0.6 & 57 & 0.8449 & 54 & 0.6592 \\
\hline
3 & 0.65 & 105 & 0.8472 & 67 & 0.6582 \\
\hline
3 & 0.7 & 182 & 0.8531 & 89 & 0.7153 \\
\hline
3 & 0.75 & 192 & 0.8542 & 169 & 0.7148 \\
\hline
3 & 0.8 & 211 & 0.8766 & 198 & 0.717 \\
\hline
3 & 0.85 & 288 & 0.8786 & 272 & 0.7398 \\
\hline
3 & 0.9 & 288 & 0.8786 & 377 & 0.7871 \\
\hline
3 & 0.95 & 342 & 0.8855 & 416 & 0.799 \\
\hline
3 & 1.0 & 934 & 0.8984 & 934 & 0.8088 \\
\hline
4 & 0.4 & 70 & 0.8341 & 64 & 0.6998 \\
\hline
4 & 0.45 & 70 & 0.8341 & 64 & 0.6998 \\
\hline
4 & 0.5 & 83 & 0.8525 & 64 & 0.6998 \\
\hline
4 & 0.55 & 83 & 0.8525 & 76 & 0.7145 \\
\hline
4 & 0.6 & 83 & 0.8525 & 76 & 0.7145 \\
\hline
4 & 0.65 & 128 & 0.8545 & 89 & 0.7135 \\
\hline
4 & 0.7 & 204 & 0.8605 & 108 & 0.741 \\
\hline
4 & 0.75 & 211 & 0.8616 & 185 & 0.7334 \\
\hline
4 & 0.8 & 230 & 0.884 & 211 & 0.7359 \\
\hline
4 & 0.85 & 307 & 0.8859 & 281 & 0.75 \\
\hline
4 & 0.9 & 307 & 0.8859 & 384 & 0.7878 \\
\hline
4 & 0.95 & 358 & 0.8893 & 422 & 0.7997 \\
\hline
4 & 1.0 & 934 & 0.8984 & 934 & 0.8088 \\
\hline
5 & 0.4 & 128 & 0.8564 & 128 & 0.7516 \\
\hline
5 & 0.45 & 128 & 0.8564 & 128 & 0.7516 \\
\hline
5 & 0.5 & 140 & 0.8748 & 128 & 0.7516 \\
\hline
5 & 0.55 & 140 & 0.8748 & 128 & 0.7516 \\
\hline
5 & 0.6 & 140 & 0.8748 & 128 & 0.7516 \\
\hline
5 & 0.65 & 179 & 0.8772 & 140 & 0.7507 \\
\hline
5 & 0.7 & 256 & 0.8832 & 153 & 0.7643 \\
\hline
5 & 0.75 & 256 & 0.8832 & 230 & 0.7567 \\
\hline
5 & 0.8 & 268 & 0.8878 & 256 & 0.7592 \\
\hline
5 & 0.85 & 345 & 0.8897 & 320 & 0.7723 \\
\hline
5 & 0.9 & 345 & 0.8897 & 409 & 0.7975 \\
\hline
5 & 0.95 & 396 & 0.8931 & 435 & 0.8001 \\
\hline
5 & 1.0 & 934 & 0.8984 & 934 & 0.8088 \\
\hline
6 & 0.4 & 256 & 0.8829 & 256 & 0.7808 \\
\hline
6 & 0.45 & 256 & 0.8829 & 256 & 0.7808 \\
\hline
6 & 0.5 & 256 & 0.8829 & 256 & 0.7808 \\
\hline
6 & 0.55 & 256 & 0.8829 & 256 & 0.7808 \\
\hline
6 & 0.6 & 256 & 0.8829 & 256 & 0.7808 \\
\hline
6 & 0.65 & 281 & 0.8869 & 256 & 0.7808 \\
\hline
6 & 0.7 & 358 & 0.8929 & 256 & 0.7808 \\
\hline
6 & 0.75 & 358 & 0.8929 & 332 & 0.7733 \\
\hline
6 & 0.8 & 358 & 0.8929 & 358 & 0.7758 \\
\hline
6 & 0.85 & 435 & 0.8948 & 409 & 0.781 \\
\hline
6 & 0.9 & 435 & 0.8948 & 486 & 0.8048 \\
\hline
6 & 0.95 & 486 & 0.8982 & 512 & 0.8074 \\
\hline
6 & 1.0 & 1024 & 0.8984 & 1024 & 0.8087 \\
\hline
7 & 0.4 & 512 & 0.889 & 512 & 0.7869 \\
\hline
7 & 0.45 & 512 & 0.889 & 512 & 0.7869 \\
\hline
7 & 0.5 & 512 & 0.889 & 512 & 0.7869 \\
\hline
7 & 0.55 & 512 & 0.889 & 512 & 0.7869 \\
\hline
7 & 0.6 & 512 & 0.889 & 512 & 0.7869 \\
\hline
7 & 0.65 & 512 & 0.889 & 563 & 0.8068 \\
\hline
7 & 0.7 & 563 & 0.8983 & 614 & 0.8031 \\
\hline
7 & 0.75 & 563 & 0.8983 & 614 & 0.8031 \\
\hline
7 & 0.8 & 563 & 0.8983 & 614 & 0.8031 \\
\hline
7 & 0.85 & 614 & 0.8949 & 665 & 0.8083 \\
\hline
7 & 0.9 & 614 & 0.8949 & 665 & 0.8083 \\
\hline
7 & 0.95 & 665 & 0.8983 & 665 & 0.8083 \\
\hline
7 & 1.0 & 1024 & 0.8984 & 1024 & 0.8087 \\
\hline
8 & 0.4 & 1024 & 0.8984 & 1024 & 0.8087 \\
\hline
8 & 0.45 & 1024 & 0.8984 & 1024 & 0.8087 \\
\hline
8 & 0.5 & 1024 & 0.8984 & 1024 & 0.8087 \\
\hline
8 & 0.55 & 1024 & 0.8984 & 1024 & 0.8087 \\
\hline
8 & 0.6 & 1024 & 0.8984 & 1024 & 0.8087 \\
\hline
8 & 0.65 & 1024 & 0.8984 & 1024 & 0.8087 \\
\hline
8 & 0.7 & 1024 & 0.8984 & 1024 & 0.8087 \\
\hline
8 & 0.75 & 1024 & 0.8984 & 1024 & 0.8087 \\
\hline
8 & 0.8 & 1024 & 0.8984 & 1024 & 0.8087 \\
\hline
8 & 0.85 & 1024 & 0.8984 & 1024 & 0.8087 \\
\hline
8 & 0.9 & 1024 & 0.8984 & 1024 & 0.8087 \\
\hline
8 & 0.95 & 1024 & 0.8984 & 1024 & 0.8087 \\
\hline
8 & 1.0 & 1024 & 0.8984 & 1024 & 0.8087 \\
\hline

\end{longtable}
\end{center}

\end{document}