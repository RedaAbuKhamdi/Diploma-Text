\section{Введение}
Компьютерная томография (КТ) является одним из ключевых методов неразрушающего контроля и широко применяется как в медицинской диагностике, так и в промышленности. Её основной принцип заключается в восстановлении внутренней структуры объекта по множеству его проекций, полученных под различными углами.

В промышленной томографии КТ используется для анализа геометрии, выявления скрытых дефектов, построения CAD-моделей и контроля соответствия изделий стандартам качества. 

При этом одной из важнейших задач остаётся снижение дозовой нагрузки и времени эксперимента, особенно в случаях, когда объект чувствителен к излучению или необходима высокая скорость сканирования.

Одним из современных направлений развития является применение протоколов мониторинговой реконструкции, при которых процесс сканирования и реконструкции выполняется итеративно, с возможностью анализа промежуточных результатов. Такой подход призван минимизировать дозу излучения и сократить время эксперимента путем предоставления возможности остановки процесса сканирования если достигнута достаточное качество реконтсрукции с имеющимися проекциями.

Отметим, крайнюю прикладную актуальность рассматриваемой задачи, как в сфере медицины, так и инженерии, и электроники. В медицине данная проблема возникает сравнительно часто при исследовании поврежденных тканей. В частности, при КТ и рентгенографии опухолей, различной этиологии (в особенности носящих агрессивный характер), при исследовании некротических поражений, при дифференциальной диагностике осложнений и новообразований в высокочувствительных к радиации органах. В научной периодике, сравнительно часто встречаются различные эвристические и не апробированные подходы по решению данной задачи. В настоящее время существует всего одно исследование, где было выработано решение не общей постановки задачи в рамках абстрактного улучшения качества, а решена конкретная проблема - определены и обоснованы характеристики процедуры компьютерной томографии для снижения дозы радиации при детектировании COVID-19 в легких \cite{bulatov2023reducing}. 

Для инженерии и электротехники потребность в развитии протоколов мониторинговой реконструкции в компьютерной томографии прослеживается на этапе контроля качества производства оборудования и высокотехнологичных компонентов, где излишнее радиационное воздействие может оказать негативный эффект (и при этом использование иных методов неразрушающего контроля не позволит гарантировать бездефектность исследуемого объекта). Например, подобные задачи возникают при контроле микроэлектроники, элементов и узлов спутников, электронных микроскопов, высокоточной оптики. 

Задача бинаризации является одной из типовых задач в индустриальной томографии, и ее решение позволяет отбросить артефакты реконструкции. Это в свою очередь делает возможным построение меш и CAD модели исследуемых объектов, выполнение контроля производства и автоматического обнаружения дефектов.

На этапе разработки метода компьютерной томографии в 1970 – 1980 -гг., подобные идеи являлись заделом на будущее. В то время основные задачи концентрировались в части развития самого метода томографии. Теперь, с развитием компьютерной техники и программного обеспечения, и, в частности, с возникновением протоколов мониторинговой реконструкции появилась возможность разработки правила останова, для определения минимально необходимого набора углов для получения удовлетворительной бинарной сегментации реконструкций. 

При этом, следует отметить, что метрики, правила останова и реализация протокола могут существенно отличаться в зависимости от решаемой задачи. 

Вместе с тем, получение результатов по различным объектам и последующее их обобщение могут привести к разработке универсальных и общих зависимостей, что существенно повысит управляемость и качество процедуры томографии под контролем реконструкции. 

Актуальность представленных проблем, стала основой выработки цели данного исследования.
Цель данной работы - Разработка и исследование протоколов мониторинговой реконструкции в рамках задачи бинаризации.
Для достижения цели был сформирован следующий комплекс задач:
\begin{enumerate}
    \item Изучить текущую научную периодику по проблеме бинарной сегментации и метриках оценки их качества.
    \item Выбрать алгоритмы бинаризации, подходящие для практического применения в anytime процессе мониторинговой реконструкции.
    \item Подобрать и сгенерировать исходные данные для исследования поведения бинаризаций реконструкций в томографии под контролем реконструкции.
    \item Сформировать набор метрик для оценки качества бинарных сегментаций реконструкций.
    \item Разработать программный комплекс конвейера экспериментов симулирующих процесс томографии под контролем реконструкции.
    \item Сформулировать правило останова
    \item С помощью реализованного программного комплекса провести численные эксперименты по исследованию эффективности правила останова.
    \item Обработать полученные результаты и установить эффект правила останова на количество необходимых проекций для получения качественной бинаризации реконструкции
\end{enumerate}
Предмет работы: разработка правила останова для определения минимального необходимого набора углов для получения качественной бинаризации реконструкции.
Объект работы: процесс томографии под контролем реконструкции и процедура бинарной сегментации полученных реконструкций.
