\section{Введение}

Компьютерная томография (КТ) является одним из ключевых методов неразрушающего контроля и широко применяется как в медицинской диагностике, так и в промышленности. Основной принцип КТ заключается в восстановлении внутренней структуры объекта по множеству его проекций, полученных под различными углами.

В промышленной томографии метод используется для анализа геометрии, выявления скрытых дефектов, построения CAD-моделей и контроля соответствия изделий стандартам качества. Одной из важнейших задач при этом остаётся снижение дозовой нагрузки и времени проведения эксперимента, особенно в случаях, когда объект чувствителен к излучению или когда процесс сканирования занимает достаточно длительное время.

Современным направлением развития в этой области является применение протоколов мониторинговой реконструкции, при которых процесс сканирования и реконструкции выполняется итеративно с возможностью анализа промежуточных результатов. Такой подход делает процесс сканирования anytime-алгоритмом так как позволяет остановить сканирование при достижении достаточного качества реконструкции с уже имеющимися проекциями, потенциально уменьшая дозу излучения и время сканирования.

Задача снижения дозы является актуальной как в медицине, так и в инженерии и микроэлектронике. В медицине она возникает, например, при исследовании онкологических заболеваний, некротических поражений и других патологий в органах, чувствительных к радиации. При этом в научной литературе часто рассматриваются частные или эвристические подходы. На данный момент существует лишь одно исследование, в котором протокол мониторинговой реконструкции был применён к конкретной задаче — снижению дозовой нагрузки при обнаружении COVID-19 в лёгких на КТ-изображениях \cite{bulatov2023reducing}.

В инженерии и электронике потребность в подобных протоколах особенно выражена при контроле высокоточных и чувствительных изделий — например, в микроэлектронике, оптоэлектронике и спутниковых технологиях, где применение других методов контроля не всегда позволяет гарантировать надёжность.

Одной из базовых задач в индустриальной томографии является задача бинаризации реконструкций. Её решение позволяет отбросить артефакты, сформировать CAD-модели объектов, автоматизировать поиск дефектов и выполнить контроль соответствия изделий требованиям. Задача бинаризации имеет важное прикладное значение, и точность её решения напрямую влияет на результаты последующей обработки.

На ранних этапах развития томографии (1970–1980-е гг.) подобные задачи воспринимались как перспективные направления, поскольку внимание было сосредоточено на разработке самого метода. С развитием вычислительных средств и появлением протоколов мониторинговой реконструкции появилась возможность перехода к более гибким стратегиям, включая построение правила останова — критерия, определяющего момент завершения сбора данных, когда достигнута реконструкция удовлетворительного качества.

Следует отметить, что правила останова, выбор метрик качества и архитектура протокола могут существенно различаться в зависимости от поставленной задачи. Однако, систематическое исследование таких протоколов в рамках конкретных задач — в частности, бинаризации — остаётся недостаточно проработанным.

Актуальность указанных проблем определяет цель настоящего исследования. 

\textbf{Цель работы} — разработка и исследование протоколов мониторинговой реконструкции в рамках задачи бинаризации.

Для достижения поставленной цели были поставлены следующие задачи:

\begin{enumerate}
    \item Изучение текущего состояния научных исследований по вопросам бинарной сегментации и метрик её оценки.
    \item Выбор алгоритмов бинаризации, подходящих для практического применения в условиях мониторинговой реконструкции.
    \item Подбор и генерация набора изображений для исследования поведения бинаризации в процессе томографии под контролем реконструкции.
    \item Формирование набора метрик оценки качества бинарных сегментаций.
    \item Разработка программного комплекса, реализующего конвейер экспериментов, симулирующий процесс томографии под контролем реконструкции с применением бинаризации на каждой итерации.
    \item Формулировка правила останова.
    \item Проведение численных экспериментов по оценке эффективности сформулированного правила останова с помощью реализованного программного комплекса.
    \item Анализ полученных результатов и оценка влияния правила останова на количество необходимых проекций для получения качественной бинаризации реконструкции.
\end{enumerate}

\textbf{Объектом исследования} является процесс томографии под контролем реконструкции и процедура бинарной сегментации полученных реконструкций.

\textbf{Предметом исследования} является разработка правила останова, определяющего минимально необходимый набор углов сканирования для получения качественной бинарной реконструкции.
