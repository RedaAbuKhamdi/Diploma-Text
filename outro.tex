\section{Выводы}

Разработка принципов управления процедурами компьютерной томографии и их научная обоснованность одно из приоритетных направлений междисциплинарных исследований, находящихся на стыке современной прикладной математики, информатики, физики.  

Проведённое исследование протоколов мониторинговой реконструкции в рамках задачи бинаризации показывает, что применение правила останова позволяет повысить среднее качество бинаризации по исследуемым метрикам при меньшем числе углов проекций по сравнению с контролем, под которым в данном случае понимается остановка процесса на фиксированном заранее заданном наборе углов.~

В исследовании было сформулировано правило останова процедуры компьютерной томографии под контролем реконструкции операясь на анализ бинаризаций. 

Был проведен обзор корпуса литературы по алгоритмам бинаризации и метрикам оценки их качества. 

На основе проведенного обзора были выбраны алгоритмы сегментации, применимые в anytime условиях мониторинговой реконструкции. В частноти были выбраны алгоритмы Отсу, аффинного Ниблэка и классический пороговый метод.

В работы был собран набор данных, состоящий из 10 изображений, 4 из которых являются сканами реальных объектов, а остальные 6 синтетические.

Изображения основанные на сканов объектов включают 2 изображения реконструкции бетонной плиты и две трехмерные модели объектов, полученные с помощью сканнера - фигурка кролика и тайская статуэтка.

Синтетические изображения были сгенерированы с разнообразной геометрией, чтобы симулировать различные ситуации в мониториновой реконструкции. 

Эти синтетические изображения разбиты на 3 пары. 

Первая такая пара - ортогональная и наклонная сетки, симулирующие изображения, которые требуют небольшое количество углов для получения достаточного уровня детализации в реконструкции.

Следующей парой являются гладкие объекты - трехмерная гауссиана и набор эллипсов. В силу гладкости этих объектов ожидается, что добавление большего количества углов проекций будет плавно повышать качество реконструкций.

Последней парой синететических объектов являются полигоны. Эти изображения состоят из двух пересекающихся пирамид и одного шестиугольника. Предполагается, что в силу наличия прямых линий и острвх углов, при включении определенного набора проекций должны быть дискретные скачки в качестве.
Количество углов полностью покрывающих все геометрические особенности этих изображений значительно больше, чем в случае с сетками.

Далее был сформирован набор метрик на основе результатов численных экспериментов на модельных изображениях, симулирующих различные виды ошибок. Такими метриками, чувствительными ко всем видам ошибок, определенных в эксперименте, является IOU и Symmetric Boundary DICE.

В работы был спроектирован и разработан программный комплекс для конвейера экспериментов, включающий в себя модуль реконструкции, бинаризации, оценки бинаризаций и модуль визуализации результатов. Реализация программного комплекса выполнена на языке Python с использованием библиотек numpy, cupy, matplotlib и astra toolbox.

Далее было сформировано правило останова, настраиваемое двумя параметрами - \(alpha\) и \(c\), первый из которых регулирует минимальный индекс набора углов для начала действия правила останова, а второй - порох схожести соседних бинаризаций.

На основе правила останова была разработана методика для определения его эффективности. Осуществлен комплекс численных экспериментов, позволяющий установить эффективность разработанного правила.

На основе программного комплекса и разработанной методики запущены эксперменты и получены результаты, указывающие на значительный выигрыш в среднем качестве бинаризаций по обеим метрикам по сравнению с контролем для алгоритмов Отсу и классического порогового. Применение правила останова для алгоритма аффинного Ниблэка показало незначительный, хотя и положительный, результат. Вероятно это связано с спецификой самого алгоритма в большей степени, чем с правилом останова.

Среди наиболее важных результатов работы необходимо выделить следующие:
\begin{enumerate}
    \item Предложено правило останова на основе анализа качества бинаризаций реконструкций.
    \item Сформирована методологическая и программная база для проведения экспериментов по исследованию эффективности правила останова.
    \item Проведены численные эксперименты по оценки эффективности правила останова.
    \item Полученны положительные результаты при применении предложенного правила останова.
    \item Установлено, что использование бинаризации является допустимым и перспективным подходом к автоматическому определению точки останова в мониторинговой реконструкции.
\end{enumerate}

Полученные результаты создают предпосылки для проведения ряда последующих исследований. В частности, представляет интерес изучение выбора параметров правила останова для различных задач и выявление закономерностей их влияния на качество результата. Кроме того, важным направлением является экспериментальная проверка предложенного правила останова в практических условиях на КТ-сканировании. Перспективным также является исследование других быстрых алгоритмов бинаризации и метрик, пригодных для применения в условиях anytime-реконструкции. В дальнейшем значимым представляется и проведение работы по модификации правила останова с целью построения более универсального подхода.