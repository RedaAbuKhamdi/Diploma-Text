\section{Выводы}

Разработка принципов управления процедурами компьютерной томографии и их научная обоснованность - одно из приоритетных направлений междисциплинарных исследований, находящихся на стыке современной прикладной математики, информатики, физики.  

Проведённое исследование протоколов мониторинговой реконструкции в рамках задачи бинаризации показывает, что применение правила останова позволяет повысить среднее качество бинаризации по исследуемым метрикам при меньшем числе углов проекций по сравнению с контролем, под которым в данном случае понимается остановка процесса на фиксированном заранее заданном наборе углов.~

В исследовании было сформулировано правило останова процедуры компьютерной томографии под контролем реконструкции опираясь на анализ бинаризаций. 

В работы был собран набор данных, состоящий из десяти изображений, четыри из которых являются сканированными изображениями реальных объектов, а остальные шесть синтетические.

В исследовании использовались метрики IOU и Symmetric Boundary DICE, а также алгоритмы бинаризации: Отсу, классический пороговый и аффинный Ниблэк.

В работе был спроектирован и разработан программный комплекс для конвейера экспериментов, включающий в себя модули реконструкции, бинаризации, оценки бинаризаций и модуль визуализации результатов. Реализация программного комплекса выполнена на языке Python с использованием библиотек numpy, cupy, matplotlib и astra toolbox.

Было сформировано правило останова, настраиваемое двумя параметрами - \(alpha\) и \(c\), первый из которых регулирует минимальный индекс набора углов, для начала действия правила останова, а второй - порог схожести соседних бинаризаций.

Осуществлен комплекс численных экспериментов, позволяющий установить эффективность разработанного правила.

В результате экспериментов было установлено, что применение правила останова позволяет достичь качества, сравнимого с использованием полного набора углов, в среднем за вдвое меньшее количество проекций.

Разница в качестве при этом, в среднем по всем алгоритмам, не привышает 0.02 едриницы по всем метрикам.

Результаты работы показывают, что применение правила останова в томографии под контролем реконструкции позволяет завершить процесс при меньшем числе углов проекций. Это, в свою очередь, способствует снижению дозы излучения и сокращению времени, необходимого для проведения КТ-исследования.

Полученные результаты открывают направления для дальнейших исследований, включая оптимизацию параметров правила останова, его проверку в практических условиях КТ-сканирования и адаптацию под различные типы данных. Перспективным также является изучение альтернативных алгоритмов бинаризации и метрик для применения в условиях anytime-реконструкции.